\begin{minipage}[t]{.5\linewidth}

	\subsubsection*{1. Introduction}

	\subsubsection*{2. Syst�mes GNU/Linux}
	\begin{itemize}
		\item Pr�sentation de Linux
		\item Le noyau Linux
		\item M�thode d'installation 
		\item Syst�mes de fichiers 
		\item Partitionnement et gestion des disques
		\item Packages 
		\item Configuration r�seau et r�glages de base
	\end{itemize}

	\subsubsection*{3. Focus sur Debian GNU/Linux}
	\begin{itemize}
		\item M�thodes d'installation
		\item Installation et r�glages de base
		\item Syst�me de packages Debian
	\end{itemize}

	\subsubsection*{4. TP : Installation Linux Debian}

        \subsubsection*{5. Administration syst�me et s�curit�}
        \begin{itemize}
                \item S�curit� physique et au d�marrage
                \item Authentification
                \item Quotas et gestion des droits
                \item Gestion de l'authentification
                \item Droits sur les applications
                \item S�curit� applicative
                \item OpenSSH
                \item Transfert de fichiers
                \item Isoler les services
                \item Gestion des journaux et monitoring
                \item Retour sur le d�marrage : systemd vs init.d en Debian 8
		\item Crontab
                \item Proc�dures de sauvegarde
        \end{itemize}

        \subsubsection*{6. TP : Administration syst�me}

\end{minipage} 
\hfill 
\begin{minipage}[t]{.5\linewidth}

	 \subsubsection*{7. Shell et associ�}
        \begin{itemize}
		\item L'�diteur Vim
		\item Expressions rationnelles
                \item Commandes shell "internes"
                \item Commandes externes principales
                \item Configuration du shell (bashrc, alias, etc.)
                \item Crontab et redirections entr�e/sorties
                \item Script shell
        \end{itemize}

	\subsubsection*{8. TP : Shell et scripting}

\end{minipage}
