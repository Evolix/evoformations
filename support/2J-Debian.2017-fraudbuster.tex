%%%%%%%%%%%%%%%%%%%%%%%%%%%%%%%%%%%%%%%%%%%%%%%%%%%%%%%%
% Copyright (c) 2004-2011 eVoLiX. Tous droits reserves.%
%%%%%%%%%%%%%%%%%%%%%%%%%%%%%%%%%%%%%%%%%%%%%%%%%%%%%%%%

\documentclass[11pt,a4paper,oneside]{report}

%% Extensions
% \usepackage[options]{extension}

% pour les liens
\usepackage{url}

% permettre accents dans le texte
\usepackage[latin1]{inputenc} % entree 8 bits iso-latin1

% prendre en compte utf8 ?
%\usepackage{ucs}
%\usepackage[utf8]{inputenc}

\usepackage[T1]{fontenc}      % encodage 8 bits des fontes utilisees

% mathematiques
\usepackage{amsmath}
\usepackage{amsfonts}
%\usepackage{euler} mathpple mathptmx
%\usepackage{euler}
%\usepackage{aeguill}
%\boldmath

% en francais
\usepackage[francais]{babel}
\usepackage[francais]{layout}
\selectlanguage{francais}

% ajout de listings, verbatim et verbatimcmd
\usepackage{listings}
\usepackage{verbatim}
\usepackage{moreverb}

% avoir de la couleur
\usepackage{color}
% definition couleurs 
\definecolor{shell}{rgb}{0.79,0.79,0.79}
\definecolor{fichier}{rgb}{0.74,0.82,0.95}
\definecolor{gris50}{rgb}{0.5,0.5,0.5}


% avoir des liens fonctionnels (avec pdflatex)
\usepackage[colorlinks=true,urlcolor=blue]{hyperref}
%\usepackage[pdftex,colorlinks=true,urlcolor=blue,pdfstartview=FitH]{hyperref}

%\pdfcompresslevel=9


%% figures
% inserer des fichiers EPS pour latex
% inserer des fichiers PDF pour pdflatex

% package pour insertion de figures
\usepackage{graphicx}
%\usepackage[dvips]{graphicx}
%\usepackage[pdftex]{graphicx}
%\DeclareGraphicsExtensions{.png, .jpg, .pdf}

% pour mettre sans serif par d�faut
\renewcommand{\familydefault}{\sfdefault}

% pour hevea
% apt-get install hevea
\usepackage{hevea}


% fonts
%\usepackage{avant}      %AvantGarde font as default 
%\usepackage{bookman}    % Bookman  
%\usepackage{chancery}   % Zapf Chancery 
%\usepackage{charter}    % Default 
%\usepackage{courier}    % Courier 
\usepackage{helvet}     % Helveticat 
%\usepackage{helvetic}   % Helvetica-Oblique 
%\usepackage{ncntrsbk}   % NewCenturySchlbk 
%\usepackage{newcent}    % NewCenturySchoolbook 
%\usepackage{palatcm}    % Palatino+Computer Modem 
%\usepackage{palatino}   % Palatino 
%\usepackage{pifont}     % Pi font support 
%\usepackage{times}      % Times 
%\usepackage{utopia}     % Utopia  
%\usepackage{zapfchan}   % ITC Zapf Chancery

% definition de caracteres
\def\tilde{\char'176}

%% Styles des pages

\textwidth=16cm
\textheight=24cm
\oddsidemargin=0pt
\topmargin=0pt
\headsep=1cm
\voffset=-3cm
\footskip=1.5cm
\headheight=2.5cm
\usepackage{fancyhdr}
\pagestyle{fancy}
\lhead{\textcolor{gris50}{\scriptsize{Administration Linux}}}
\rhead{\textcolor{gris50}{\scriptsize{Formation personnalis�e Evolix}}}

\lfoot{\scriptsize{\thepage}}
\cfoot{\textcolor{gris50}{\scriptsize{Evolix / www.evolix.fr / info@evolix.fr / 2017}}}
\rfoot{\includegraphics*[width=12mm]{evolix}}

\renewcommand{\headrulewidth}{0pt}
\renewcommand{\footrulewidth}{0.4pt}

% Variables a changer :
% TITRE, AUTEUR, SUJET et DATE

\author{Evolix}
\title{Administration Linux - Formation personnalis�e}
\date{Mai 2017}

%% debut du document

\begin{document}

\begin{center}

\vspace*{3 cm}
\Huge{{\bfseries Formation Evolix Administration Linux }} \\
~\\
\normalsize{Evolix - Mai 2017}\\

\vspace*{3 cm}
\includegraphics*[width=70mm]{pevolix}\\

\end{center}

\tableofcontents

% Copyright (c) 2004-2010 Evolix <info@evolix.fr>
%  Permission is granted to copy, distribute and/or modify this document
%  under the terms of the GNU Free Documentation License, Version 1.2
%  or any later version published by the Free Software Foundation;
%  with no Invariant Sections, no Front-Cover Texts, and no Back-Cover Texts.
%  A copy of the license is included at http://www.gcolpart.com/howto/fdl.html

\chapter{Introduction � Unix/Linux}

\section{Les d�buts}

En 1969, Ken Thompson\footnote{\url{http://www.cs.bell-labs.com/who/ken/}},
employ� dans les laboratoires Bell\footnote{\url{http://cm.bell-labs.com/}},
d�veloppe UNICS (UNiplexed Information and Computing Service), syst�me
d'exploitation mono-utilisateur �crit en langage assembleur. Rebaptis� UNIX,
Ken Thompson tenta en 1971 de r��crire le syst�me en langage FORTRAN ou en B.
Entre-temps Dennis
Ritchie\footnote{\url{http://cm.bell-labs.com/cm/cs/who/dmr/}}, �galement
employ� dans les laboratoires de Bell a pris part au projet, et mis au point
le successeur du langage B : le langage C. UNIX fut donc r��crit en langage C
et nomm� UNIX Time-Sharing System (UTS). UNIX commen�a �galement � �tre diffus�
hors des laboratoires de Bell. Or les laboratoires de Bell appartiennent � la
soci�t� AT\&T qui ne peut commercialiser autre chose que des �quipements
t�l�phoniques ou t�l�graphiques. La d�cision fut prise de distribuer le syst�me
UNIX complet avec son code source complet dans les universit�s puis �galement
dans les entreprises. De nombreuses contributions furent apport�es par
l'universit� de Berkeley (Californie) qui distribua ainsi UNIX BSD (Berkeley
Software Distribution).\\

Les laboratoires Bell d�velopp�rent le syst�me UNIX Time-Sharing System
jusqu'en octobre 1989 (sa 10�me version).  La branche commerciale d'AT\&T
d�veloppa les premi�res versions commerciales d'UNIX : System III puis System
V.\\

Les droits d'UNIX appartenant � AT\&T ont �t� rachet�s par l'entreprise
Novell\footnote{\url{http://www.novell.com/}}. En 1994, Novell a transf�r� les
droits sur la marque UNIX ainsi que les sp�cifications �
l'OpenGroup\footnote{\url{http://www.opengroup.org/}}, et a �galement revendu
le code source et l'impl�mentation UNIXWARE (d�riv�e de System V) �
l'entreprise SCO\footnote{\url{http://www.sco.com/}}.\\

L'universit� de Berkeley d�veloppa UNIX BSD jusqu'en 1993 (4.4BSD) dont
d�rivent les projets libres NetBSD, FreeBSD et OpenBSD mais �galement le
syst�me d'exploitation d'Apple MAC OS X. Du fait de la mise � disposition du
code source d'UNIX, de nombreux d�riv�s propri�taires d'UNIX furent d�velopp�s
: AIX (IBM), Solaris (Sun Microsystems), HP-UX (Hewlett-Packard), Ultrix (DEC),
Xenix (Microsoft), Unixware (SCO), Tru64 (DEC), IRIX (SGI), etc.\\

Les syst�mes d'exploitation UNIX sont multi-t�ches, multi-utilisateurs et en
g�n�ral ouverts (code source disponible).\\
~\\
\textit{Liens :} \\
\url{http://fr.wikipedia.org/wiki/UNIX} \\
\url{http://www.commentcamarche.net/unix/} \\
\url{http://www.unix.org/} \\
\url{http://www.levenez.com/unix/} \\

\section{Historique des logiciels libres}

Au commencement de l'informatique, le mat�riel �tait volumineux et co�teux. On
se focalisait surtout sur l'�quipement~: les codes des programmes �taient
disponibles gratuitement et chacun pouvait les am�liorer librement. Avec
l'apparition de la micro-informatique dans les ann�es 1980, les d�veloppements
de logiciels furent de plus en plus nombreux. De nombreuses restrictions
apparurent sur ces logiciels (licences, accord de non-divulgation) qui �taient
souvent vendus uniquement sous forme de binaires. En 1984, suite � des
probl�mes li�s � ces restrictions, Richard
Stallman\footnote{\url{http://www.stallman.org/}}, informaticien au
MIT(Institut de Technologie du Massachusetts), d�missionne du laboratoire o� il
travaille pour fonder le projet GNU\footnote{\url{http://www.gnu.org/}}. GNU
est un acronyme r�cursif qui signifie GNU's Not Unix (GNU n'est pas Unix).\\

Son ambition est de d�velopper un syst�me d'exploitation compl�tement libre et
promouvoir la libert� des logiciels. Ainsi, en 1985 est cr��e la Free Software
Fundation (FSF) qui a �crit un grand nombre de paquetages logiciels, notamment
GNU Compiler Collection (GCC) et Bourne-Again SHell
(BASH\footnote{\url{http://www.gnu.org/software/bash/}}). Des licences dites
libres ont �galement �t� r�dig�es, dont la GNU General Public License
(GPL\footnote{\url{http://www.gnu.org/copyleft/gpl.html}}) qui oblige les
programmes d�riv�s � rester avec la m�me licence.\\
~\\

D'autres licences dites libres existent comme la licence
BSD\footnote{\url{http://www.opensource.org/licenses/bsd-license.php}} qui
permet la r�utilisation du code sous la dite licence dans des programmes
propri�taires. Le terme d'Open Source d�finit un ensemble de licences moins
restrictives. L'Open Source
Initiative\footnote{\url{http://www.opensource.org/licenses/}} publie sur son
site les licences dites Open Source.
~\\ 

La d�finition de libert� pour un logiciel n'est pas forc�ment triviale et
diff�re selon certains organismes.\\
~\\
La Free Software Fundation (FSF) sp�cifie les 4 libert�s pour un logiciel libre~: \\
\begin{itemize}
  \item{La libert� d'ex�cuter le programme, pour tous les usages \textbf{(libert� 0)}.}
  \item{La libert� d'�tudier le fonctionnement du programme, et de l'adapter � vos besoins \textbf{(libert� 1)}. Pour ceci l'acc�s au code source est une condition requise.}
  \item{La libert� de redistribuer des copies, donc d'aider votre voisin, \textbf{(libert� 2)}.}
  \item{La libert� d'am�liorer le programme et de publier vos am�liorations,
pour en faire profiter toute la communaut� \textbf{(libert� 3)}. Pour ceci
l'acc�s au code source est une condition requise.}
\end{itemize}
~\\
\textit{Lien :} \url{http://www.gnu.org/philosophy/free-sw.fr.html} \\
~\\

\section{D�finitions des 'logiciels libres'}

Le projet Debian sp�cifie ses propres principes du logiciel libre~: \\

\begin{itemize}
\item{redistribution libre et gratuite}
\item{distribution du code source}
\item{aucune discrimination de personne ou de groupe}
\item{aucune discrimination de champ d'application}
\item{distribution de licences}
\item{la licence ne doit pas �tre sp�cifique � Debian}
\item{la licence ne doit pas contaminer d'autres logiciels}
\end{itemize}
~\\
\textit{Lien :} \url{http://www.debian.org/social\_contract#guidelines} \\
~\\
L'Open Source Initiative (OSI) sp�cifie les crit�res pour un logiciel Open
Source, inspir�s des principes du logiciel libre selon Debian~: \\


\begin{itemize}
  \item{Libre redistribution}
  \item{Code source}
  \item{Travaux d�riv�s}
  \item{Int�grit� du code source de l'auteur}
  \item{Aucune discrimination envers les personnes ou les groupes}
  \item{Aucune discrimination envers les champs d'effort}
  \item{Distribution de la licence}
  \item{Licence non sp�cifique d'un produit}
  \item{Licence non restrictive envers d'autres produits}
  \item{Licence neutre technologiquement}
\end{itemize}
~\\
\textit{Lien~:} \url{http://opensource.org/docs/definition.php} \\
~\\
On se rend compte que la d�finition d'un logiciel libre est beaucoup plus
complexe qu'on pourrait le croire. Elle s'appuie notamment sur la notion de
respect des droits d'auteurs, comme tout �crit, et non sur la notion de brevets
valables pour les inventions. Il faut noter que la notion de brevets logiciels
existe d�sormais dans plusieurs pays (notamment aux �tats-Unis) et des
discussions ont lieu au niveau de l'Europe pour l'adoption de ce
principe.\footnote{\url{http://brevets-logiciels.info/}}\\
~\\
Pour bien se rendre compte des subtilit�s entre les diff�rentes appellations,
il suffit d'�num�rer quelques cat�gories de logiciels~:
\footnote{\url{http://www.gnu.org/philosophy/category.fr.jpg}}\\
\begin{itemize}
\item{Logiciel libre}
\item{Logiciel Open source}
\item{Logiciel du domaine public}
\item{Logiciel copyleft� (sous gauche d'auteur)}
\item{Logiciel libre non-copyleft�}
\item{Logiciel couvert par la GPL}
\item{Logiciel priv�}
\item{Logiciel propri�taire}
\item{Shareware (Partagiciel)}
\item{Freeware}
\end{itemize}
~\\

\textit{Lien :} \url{http://www.gnu.org/licenses/license-list.fr.html}

\begin{itemize}
\item{\textbf{La licence GNU General Public License
(GPL)}\footnote{\url{http://www.gnu.org/copyleft/gpl.html}}\\ Cette licence a
�t� �crite par la FSF (Free Software Foundation) comptant pour membres Richard
Stallman et Eben Moglen\footnote{\url{http://emoglen.law.columbia.edu/}}. Elle
a �t� �crite pour fixer les conditions l�gales de distribution des logiciels du
projet GNU.\\
Elle fait partie des licences restrictives (notion de copyleft). Il faut aussi
savoir que c'est probablement la licence libre la plus utilis�e aujourd'hui
(Linux, GCC, KDE, Gnome, etc.).\\
~\\
\textit{Liens :} \\
\url{http://www.gnu.org/copyleft/gpl.html} \\
\url{http://www.linux-france.org/article/these/gpl.html}\\
\url{http://fsffrance.org/gpl/gpl.fr.html}\\
\url{http://crao.net/gpl/}\\
\item{\textbf{Les licences de type BSD}} \\
~\\
La licence BSD originale est une licence simple et permissive (elle n'est pas
soumise au principe du copyleft). N�anmoins, elle comporte une clause de
publicit� qui a provoqu� de nombreux d�bats (la licence impose un texte �
mentionner chaque fois que le logiciel est cit�). Cette clause a �t� supprim�e
par la suite dans ce qu'on appelera la licence BSD modifi�e.\\
Cette licence et des variantes de cette derni�re sont utilis�es notamment par
les projets FreeBSD, NetBSD, OpenBSD, etc.\\
Elle ne comporte aucune restriction, ce qui permet � des soci�t�s comme
Microsoft de r�utiliser le code source plac� sous cette licence, ou
d'inclure dans MacOSX des parties de FreeBSD et d'OpenBSD.\\

L'universit� de Berkeley d�veloppa UNIX BSD jusqu'en 1993 (4.4BSD) dont
d�rivent MAC OS X (Apple) ainsi que de nombreux projets libres comme NetBSD,
FreeBSD et OpenBSD, etc.\\
~\\
Le projet OpenBSD d�finit un logiciel libre, comme un logiciel sous une licence
n'apportant aucune restriction. Certaines licences peuvent �tre accept�es pour
certaines parties.\\

La capacit� � disposer d'un Unix Berkeley librement distribuable permet
d'avancer sur une base comp�titive par rapport aux autres syst�mes
d'exploitation, mais d�pend directement de la volont� des diff�rents
groupes de d�veloppement qui �changent des sources, entre-eux et entre
projets. Comprendre les implications l�gales qui entourent le concept
de "copyright" est fondamental afin de pouvoir �changer et redistribuer
du code source, et somme toute de promouvoir la coop�ration des
personnes impliqu�es.\\
~\\
Chaque syst�me se concentre sur des points particuliers, ou du moins en
th�orie :\\

\begin{itemize}
\item FreeBSD est orient� performances et applications; l'enjeu n'est
clairement pas de supporter un grand nombre d'architectures
\item NetBSD mise sur la portabilit� avant tout, avec un grand nombre
d'architectures support�es.
\item OpenBSD accentue ses efforts de d�veloppement sur la s�curit�, de
son syst�me et des applications qui le composent. Une grande attention est port�e au niveau des licences.
\end{itemize}

Il existe �galement d'autres syst�mes BSD moins connus, comme
DragonflyBSD, etc.

~\\
\textit{Lien :} \url{http://www.opensource.org/licenses/bsd-license.php} \\

\item{\textbf{La licence Artistique}} \\
Licence mise en place par le cr�ateur de Perl, Larry Wall. Outre les droits
d'utilisation, de modification, et de distribution, l'auteur conserve certains
droits (droit de n�gocier des arrangements au coup par coup, interdiction de
diffuser une version entrant en conflit avec la distribution "standard" de
l'auteur).\\
~\\
\textit{Liens :}\\
{\small
\url{http://www.opensource.org/licenses/artistic-license.php}\\
\url{http://www.perl.com/pub/language/misc/Artistic.html}\\
\url{http://linux-france.org/article/these/licence/artistic/fr-artistic.html}\\
}
\\
\item{\textbf{La GNU Lesser General Public License (LGPL)}} \\
LGPL signifie Licence publique g�n�rale limit�e GNU, ou GNU LGPL (pour GNU
Lesser General Public License) en anglais. Comme la  Licence publique g�n�rale
GNU (ou GNU GPL), elle a �t� �crite par la Free software foundation. La
diff�rence avec la GPL est que la LGPL permet de lier un programme tiers non
libre � une biblioth�que LGPL, sans pour autant r�voquer la licence (licence
non copyleft). Elle est surtout utilis�e pour des librairies (elle s'appellait
initalement Library General Public License) \\
~\\
\textit{Lien :} \url{http://www.gnu.org/licenses/lgpl.html} \\

\item{\textbf{Licence Apache}} \\ 
La licence Apache en est � sa Version 2.0 (approuv�e le 21 janvier 2004 par la
fondation Apache). Tous les projets Apache (dont Ant, Jakarta, ou Cocoon) sont
sous cette licence. Cette licence est libre, non-copyleft, et compatible GPL
(en version 1.0 et 1.1, la Apache License n'�tait pas compatible GPL).\\
~\\
\textit{Lien :} \url{http://www.apache.org/licenses/} \\

\item{\textbf{Licence X11 (ou MIT)}} \\
Cette licence est simple et permissive, sans copyleft, compatible avec la GPL
de GNU. Les anciennes versions de XFree86 utilisaient cette licence, et
quelques variantes actuelles de XFree86 l'utilisent �galement. Les licences
ult�rieures de XFree86 sont distribu�es sous la licence XFree86 1.1 (qui est
incompatible avec la GPL). La licence est parfois appel�e �licence du MIT� mais
ce terme est trompeur : le MIT a publi� ses logiciels sous diverses licences.\\
~\\
\textit{Liens :} \\
\url{http://www.x.org/Downloads\_terms.html}\\
\url{http://www.opensource.org/licenses/mit-license.php}\\

\item{\textbf{Licence Mozilla Public Licence}} \\
Cette licence n'est pas tr�s stricte en terme de "copyleft"; contrairement � la
licence X11 elle pr�sente des restrictions complexes qui la rendent
incompatibles avec la GPL de GNU. Ainsi, on ne peut pas, l�galement, lier un
module couvert par la GPL et un module couvert par la MPL. Pour entrer dans le
d�tail, la licence MPL 1.1 permet (section 13) � un programme ou � une portion
de programme d'offrir le choix entre la MPL et une autre licence. La licence
d'une partie de programme qui offre le choix de la GPL est alors compatible
avec la GPL. Les logiciels de Mozilla mais �galement NVu (cr�ation de pages
Web), Compi�re (ERP/CRM) sont sous licence MPL. SugarCRM utilise la SPL
(SugarCRM Public Licence) qui est tr�s proche de la MPL.\\
~\\
\textit{Liens :}\\
\url{http://www.mozilla.org/MPL/} \\
\url{http://www.sugarcrm.com/home/Public\_License\_FAQ/228/} \\

\item{\textbf{Licence IBM Public License}} \\
Cette licence en est � sa version 1.0 et est incompatible avec la GPL en raison
de d'exigences sp�cifiques. Elle exige notamment que certains droits soient
accord�s en-dehors de ce que la GPL pr�voit.\\
~\\
\textit{Lien :} \url{http://oss.software.ibm.com/developerworks/opensource/license10.html}} \\

\item{\textbf{Licence Sendmail \footnote{\url{http://www.sendmail.org/}}}}
~\\
Il s'agit de la licence du serveur de messagerie �lectronique �ponyme, le plus populaire sur Internet. \\
~\\
\textit{Lien :} \url{ftp://ftp.sendmail.org/pub/sendmail/LICENSE} \\

\item{\textbf{Licence Common Public License}} \\
Licence libre qui n'est pas compatible avec la GPL. La Common Public
License est incompatible avec la GPL parce qu'elle �nonce diverses
exigences sp�cifiques qui ne se trouvent pas dans la GPL. Elle exige
notamment que certaines licences de brevets soient donn�es, ce que la GPL n'exige pas. \\
~\\
\textit{Lien :} \url{http://www.eclipse.org/legal/cpl-v10.html} \\
~\\
\textbf{\textit{Licences pour un usage particulier}}
~\\
\item{\textbf{Licence GNU Free Documentation License}} \\
Cette licence a �t� con�ue pour les documents sous copyleft. Elle convient pour d'autres cat�gories d'oeuvres utiles telles que les manuels scolaires ou les dictionnaires, par exemple. Son domaine d'application n'est d'ailleurs pas exclusivement celui des oeuvres textuelles.\\
~\\
\textit{Lien :} \url{http://www.gnu.org/copyleft/fdl.html} \\

\item{\textbf{Les licences Creative Commons}} \\
Des juristes de l'universit� de Stanford ont cr�� un ensemble de licences
destin�es aux contenus tels que l'audio, la vid�o, les images, les textes et
les ressources �ducatives. Le but est de faciliter la mise � disposition et le
partage de ces contenus aux auteurs avec les restrictions qu'ils veulent (usage
commercial, possibilit� de modifications, etc.) Il suffit en effet de
visualiser deux ou trois pages illustr�es d'images explicites sur le site
Internet pour choisir la licence appropri�e. \\
~\\
\textit{Liens :} \\
\url{http://creativecommons.org/} \\
\url{http://fr.creativecommons.org/} \\
\url{http://philippe.daigremont.free.fr/CreativeCommons/BD/} \\

\item{\textbf{Licence Art Libre}} \\
Cette licence libre copyleft�e est faite pour les oeuvres artistiques. Elle
autorise la distribution commerciale, tout en pr�cisant qu'une oeuvre de plus
grande taille qui inclurait l'oeuvre soumise � la licence doit �tre elle-m�me
libre.\\
~\\
\textit{Lien :} \url{http://artlibre.org/licence.php/lal.html}\\

 
~\\
\end{itemize}

\section{Mod�le de d�veloppement}

Le mod�le de d�veloppement des logiciels libres conna�t un vif succ�s gr�ce �
l'expansion des r�seaux et d'Internet dans les ann�es 1990. Ainsi, de nombreux
projets (dont l'un des plus marquants est Linux) voient le jour avec une
communaut� r�partie dans le monde entier. Des logiciels initi�s � des buts
commerciaux basculent �galement sous des licences libres afin de profiter des
avantages de la communaut� du logiciel libre (beta-testeurs nombreux,
accroissement de l'�quipe de d�veloppeurs, politique de s�curit�, etc.). Parmi
les logiciels libres, on distingue donc de nombreux projets issus de
communaut�s (Apache, Debian, FreeBSD, NetBSD, Sendmail, etc.) et des projets
proches d'entreprises commerciales (Mandrake, Red Hat, OpenOffice, etc.). \\
~\\
En ce qui concerne l'organisation, la philosophie des logiciels libres bas�e
sur l'ouverture n'emp�che pas une structuration pr�cise. On peut ainsi d�gager
un mod�le de d�veloppement typique qui comprend~: \\
\begin{itemize}
 \item{Une ou plusieurs personnes responsables globalement du projet (souvent � l'initiative du projet ou �lues)}
 \item{Plusieurs d�veloppeurs officiels du projet ayant �t� accept�s avec pr�caution}
 \item{Des contributeurs occasionnels qui font g�n�ralement parvenir leurs contributions aux d�veloppeurs officiels}
 \item{Des utilisateurs avanc�s qui participent activement aux forums, listes de diffusion, canaux IRC}
 \item{Des utilisateurs de base qui ont � leur disposition, outre le logiciel, une documentation et des moyens d'interaction (demande de fonctionnalit�s, rapport de bogues)}
 \item{On note aussi la n�cessit� de r�dacteurs de documentation, de relecteurs, de traducteurs, de webmasters, etc.}
\end{itemize}
~\\
On constate que ce mod�le de d�veloppement n�cessite l'utilisation de nombreux
outils. Il existe des plateformes mises � la disposition des projets de
d�veloppement proposant un panel d'outils : h�bergement de site Web,
h�bergement des courriels �lectroniques, forums, listes de diffusion, outils de
d�veloppement, miroirs de t�l�chargements, etc. Parmi ces plateformes, on note
SourceForge\footnote{\url{http://www.sourceforge.net/}} (plus de 100.000
projets), FreshMeat\footnote{\url{http://freshmeat.net/}} (plus de 45.000
projets), Savannah\footnote{\url{http://savannah.gnu.org/}} (plus de 3.000
projets) ou encore Gna\footnote{\url{https://gna.org/}},
Tuxfamily\footnote{\url{http://www.tuxfamily.org/}} ou
Berlios\footnote{\url{http://www.berlios.de/}}. \\
~\\
En ce qui concerne les outils de d�veloppements collaboratifs, on note le
logiciel tr�s utilis� CVS\footnote{\url{https://www.cvshome.org/}}
(mises-�-jour contr�l�es, t�l�chargement souvent public), mais aussi des outils
plus r�cents tels que SVN\footnote{\url{http://subversion.tigris.org/}} et
Arch\footnote{\url{http://www.gnu.org/software/gnu-arch/}} ou encore
GIT\footnote{\url{http://git.or.cz/}}. Pour la gestion des outils de
communication, on note les outils de publication web
(SPIP\footnote{\url{http://www.spip.net/}},
Wiki\footnote{\url{http://en.wikipedia.org/wiki/Wiki}}), les forums
(PHPBB\footnote{\url{http://www.phpbb.com/}},
Phorum\footnote{\url{http://phorum.org/}}), les gestionnaires de listes de
diffusion (Sympa\footnote{\url{http://www.sympa.org/}},
Mailman\footnote{\url{http://www.gnu.org/software/mailman/}}), les serveurs IRC
(Freenode\footnote{\url{http://freenode.net/}},
Undernet\footnote{\url{http://www.undernet.org/}}), etc. \\
~\\
La connaissance de ces outils permet de mieux appr�hender la diversit� des
membres de la communaut� du logiciel libre. Des r�gles sont implicites aux
utilisateurs finaux, � savoir lire la documentation officielle, les FAQ (Foires
Aux Questions) ainsi que les archives des forums ou listes de diffusion avant
d'utiliser les outils de communication. On se r�f�rera aux documents "Les
r�gles de la
Netiquette"\footnote{\url{http://www.sri.ucl.ac.be/SRI/rfc1855.fr.html}},
"Comment poser les questions de mani�re
intelligente"\footnote{\url{http://www.gnurou.org/documents/smart-questions-fr.html}},
"Comment signaler efficacement un
bug"\footnote{\url{http://www.chiark.greenend.org.uk/~sgtatham/bugs-fr.html}}
ou encore "Comment faire un rapport sans se faire
lyncher"\footnote{\url{http://www.asktog.com/columns/047HowToWriteAReport.html}}.
Il faut bien noter l'organisation souvent pyramidale du mod�le de
d�veloppement. Ainsi, m�me s'il est souvent possible de contacter directement
les responsables d'un projet par courrier �lectronique, cela se fera avec
r�serve et uniquement pour des questions majeures. Il semble aussi entendu que
pour obtenir des droits ou des responsabilit�s dans un projet, les �chelons se
gravissent petit � petit.\\
~\\
Les logiciels en eux-m�mes, produits finaux des �quipes de d�veloppement, sont
souvent disponibles sous plusieurs formes. G�n�ralement, on distingue une
version dite stable et une version en cours de d�veloppement. La version stable
est une version ayant subi plusieurs phases de tests et corrections. C'est
cette version que le projet propose d'installer aux utilisateurs de base et
c'est encore plus vrai en environnement professionnel.\\
La version en cours de d�veloppement (parfois accessible � partir des outils de
d�veloppements collaboratifs comme CVS) est � r�server aux d�veloppeurs, aux
utilisateurs d�sirant contribuer en rapportant les erreurs, ou bien aux
impatients.\\
Il faut noter que les contraintes impos�es � une version stable diff�rent d'un
projet � un autre. Ce sera �galement le cas avec les num�ros de version, qui
n'ont plus grande signification du fait des politiques de num�rotation
diff�rentes entre les projets. � l'exception des mentions "Version Beta",
"Realease Candidate"... qui signifient qu'il s'agit de versions en cours de
correction et validation avant une sortie officiellement stable. Sans que cela
soit une r�gle absolue, on peut n�anmoins distinguer certaines r�gles communes,
voir par exemple une tentative de formalisation de certaines
conventions\footnote{\url{http://semver.org/}}.\\
~\\
Les utilisateurs de logiciels libres prennent donc part au mod�le de
d�veloppement gr�ce au support communautaire qui leur permet non seulement de
trouver de l'aide mais de rapporter les erreurs �ventuellement rencontr�es et
de demander l'ajout de nouvelles fonctionnalit�s. Parmi les utilisateurs de
logiciels, on trouve �galement de nombreuses structures professionnelles.
Ainsi, les contraintes engendr�es par l'utilisation dans un environnement
professionnel ont n�cessit� la cr�ation de support commercial pour certains
logiciels libres. De m�me, on constate l'apparition de structures sp�cialis�es
dans le support de solutions libres.\\
En France, il existe ainsi de nombreuses SS2L (Soci�t� de Service en Logiciels
Libres) r�parties sur le territoire. D'une mani�re g�n�rale, l'utilisation des
logiciels libres dans le monde professionnel est un ph�nom�ne en vogue
actuellement ; il en r�sulte souvent de nombreux avantages (fiabilit� accrue,
correction de bogues, etc.) pour les logiciels libres concern�s. \\
~\\
\textit{Liens :} \\
\url{http://www.gnu.org/prep/SERVICE} \\
\url{http://www.linux-france.org/article/pro/annuaire/} \\


% Copyright (c) 2004-2010 Evolix <info@evolix.fr>
%  Permission is granted to copy, distribute and/or modify this document
%  under the terms of the GNU Free Documentation License, Version 1.2
%  or any later version published by the Free Software Foundation;
%  with no Invariant Sections, no Front-Cover Texts, and no Back-Cover Texts.
%  A copy of the license is included at http://www.gcolpart.com/howto/fdl.html

\chapter{Syst�mes GNU/Linux}

\section{Pr�sentation de Linux}

Linux est un noyau de syst�me d'exploitation de type UNIX cr�� par Linus
Torvalds et de nombreux d�veloppeurs. \\

Tout ordinateur inclue un ensemble basique de programmes appel� syst�me
d'exploitation. Le programme le plus important d'un syst�me d'exploitation est
appel� le noyau : il est charg� dans la m�moire physique (RAM) quand le syst�me
d�marre et contient les instructions n�cessaires � l'ordinateur pour
fonctionner. Les autres programmes permettent d'interagir avec l'ordinateur
mais ils sont moins importants car les possibilit�s d'interaction avec
l'ordinateur sont d�termin�es par le noyau. \\

Le 5 octobre 1991, Linus Torvalds, un informaticien finlandais, annonce sur un
forum Usenet la disponibilit� du syst�me d'exploitation Linux, inspir� de
Minix.

\begin{verbatim}
From: Linus Benedict Torvalds (torvalds@klaava.Helsinki.FI)
Subject: Free minix-like kernel sources for 386-AT
Newsgroups: comp.os.minix
Date: 1991-10-05 08:53:28 PST

Do you pine for the nice days of minix-1.1, when men were men and wrote
their own device drivers? Are you without a nice project and just dying
to cut your teeth on a OS you can try to modify for your needs? Are you
finding it frustrating when everything works on minix? No more all-
nighters to get a nifty program working? Then this post might be just
for you

As I mentioned a month ago, I'm working on a free version of a
minix-lookalike for AT-386 computers. It has finally reached the stage
where it's even usable (though may not be depending on what you want),
and I am willing to put out the sources for wider distribution. It is
just version 0.02 (+1 (very small) patch already), but I've successfully
run bash/gcc/gnu-make/gnu-sed/compress etc under it. 

Sources for this pet project of mine can be found at nic.funet.fi
(128.214.6.100) in the directory /pub/OS/Linux. The directory also
contains some README-file and a couple of binaries to work under linux
(bash, update and gcc, what more can you ask for . Full kernel
source is provided, as no minix code has been used. Library sources are
only partially free, so that cannot be distributed currently. The
system is able to compile "as-is" and has been known to work. Heh.
Sources to the binaries (bash and gcc) can be found at the same place in
/pub/gnu. 

[...]

I can (well, almost) hear you asking yourselves "why?". Hurd will be
out in a year (or two, or next month, who knows), and I've already got
minix. This is a program for hackers by a hacker. I've enjouyed doing
it, and somebody might enjoy looking at it and even modifying it for
their own needs. It is still small enough to understand, use and
modify, and I'm looking forward to any comments you might have.

I'm also interested in hearing from anybody who has written any of the
utilities/library functions for minix. If your efforts are freely
distributable (under copyright or even public domain), I'd like to hear
from you, so I can add them to the system. I'm using Earl Chews estdio
right now (thanks for a nice and working system Earl), and similar works
will be very wellcome. Your (C)'s will of course be left intact. Drop me
a line if you are willing to let me use your code.

Linus
\end{verbatim}

Linux est multi-t�ches, multi-utilisateurs et compatible Unix (il respecte les
normes POSIX). Con�u au d�part pour les machines de type Intel x86, Linux est
maintenant disponible pour les architectures PowerPC, Alpha, MIPS, Sparc, etc. \\

Les sources de Linux sont disponibles (sous licence GPL) et l'explosion
d'Internet a permis un mode de d�veloppement innovant~: de nombreux
d�veloppeurs (b�n�voles ou r�mun�r�s par des entreprises) participent au
d�veloppement de Linux et forment avec tous les utilisateurs une communaut�.
Linux est d'ailleurs l'un des exemples les plus connus de logiciel libre.\\

L'un des objectifs du projet GNU est d'avoir un syst�me d'exploitation
compl�tement libre. Or, de tr�s nombreux outils ont �t� d�velopp�s par le
projet GNU (GCC, Bash, etc.) mais Hurd, noyau d�velopp� par le projet GNU,
tardant � sortir, Linux a �t� adopt� par le projet GNU en 1993 pour �tre le
noyau du syst�me pr�n� par le projet GNU. L'ensemble form� par les outils du
projet GNU et du noyau Linux est souvent appel� syst�me GNU/Linux.\\

Une distribution est un syst�me GNU/Linux avec un certain de nombre de choix et
d'outils mis � disposition pour g�rer au mieux les logiciels et leurs
configurations. Il existe un grand nombre de distributions adapt�es � un usage
(ou un mat�riel) sp�cifique. Les distributions g�n�ralistes les plus connues
sont Red Hat, Debian, Mandriva, SuSe, Gentoo, Slackware, Fedora.

\section{M�thode d'installation}

Il existe diverses m�thodes d'installation~: \\

Dans les ann�es 1990, on installait un syst�me GNU/Linux � l'aide de
(nombreuses) disquettes. Aujourd'hui la m�thode d'installation la plus r�pandue
est d'utiliser un jeu de CD-ROM ou un DVD-ROM t�l�charg� sur Internet par HTTP ou
FTP. D'autres m�thodes existent selon les distributions et peuvent s'av�rer
tr�s pratiques notamment des installations amorc�es par le r�seau.~\\

En cas de probl�me � l'installation d'un syst�me GNU/Linux, voici quelques suggestions~: \\
~\\
\begin{itemize}
\item{Prendre garde � la fiabilit� des p�riph�riques d'amor�age (CD-ROM, DVD-ROM) en v�rifiant syst�matiquement l'empreinte MD5 des CD-ROMs}\\
\item{Prendre connaissance des param�tres sp�cifiques d'amor�age du noyau Linux permettant d'�viter certains dysfonctionnements.}\\
\item{Pour les controleurs RAID ou cartes r�seau exotiques (ou r�centes), il faudra peut-�tre charger un module sp�cifique lors de l'installation.} \\
\item{V�rifier la compatibilit� du mat�riel. Il existe plusieurs sites dont le "HardWare Howto"\footnote{\url{http://www.tldp.org/HOWTO/Hardware-HOWTO/}} ou encore le "Debian GNU/Linux device driver check page"\footnote{\url{http://kmuto.jp/debian/hcl/}}}\\
\item{Utiliser les moyens de support communautaire tels que les listes de diffusion ou les canaux IRC. En cas de bogue (ou pas) avec l'installation de Debian, il faut rapporter le bogue pour le pseudo-paquet "installation-reports"\footnote{\url{http://www.debian.org/devel/debian-installer/report-template}}}\\
\item{Un support professionnel peut �galement �tre trouv� aupr�s de soci�t�s de services en logiciels libres.}
\end{itemize}
~\\
\textit{Liens~:} \\
\url{http://www.tldp.org/HOWTO/BootPrompt-HOWTO.html}\\
\url{http://people.debian.org/~blade/install/preload/index.old.html}\\
\url{http://www.gnu.org/prep/service.html}\\

\section{Syst�mes de fichiers}

Il existe une norme appel�e Filesystem Hierarchy Standard\footnote{\url{http://www.pathname.com/fhs/}} � laquelle se conforment certaines distributions. Rappelons la hi�rarchie du syst�me de fichiers (cela varie d'une distribution � une autre) d'un syst�me GNU/Linux~: \\
~\\
\begin{center}
\begin{tabular}{|c|c|}
\hline
Arborescence & Contenu \\
\hline
 bin   &    Binaires (ex�cutables) des commandes essentielles \\
\hline
 boot  &    Fichiers statiques pour le chargeur d'amor�age (boot)  \\
\hline
dev   &    Fichiers des pilotes de p�riph�riques \\
\hline
etc  &     Configuration syst�me propre � la machine \\
\hline
home  &    R�pertoires personnels des utilisateurs \\
\hline
lib    &   Biblioth�ques partag�es et modules noyaux essentiels \\
\hline
mnt,media & Points de montage pour les montages temporaires \\
\hline
proc,sys & R�pertoire virtuel pour les informations syst�mes \\
\hline
root  &    R�pertoire personnel de l'utilisateur root \\
\hline
sbin   &   Ex�cutables syst�me essentiels \\
\hline
tmp  &     Fichiers temporaires \\
\hline
usr   &    Hi�rarchie secondaire \\
\hline
var  &     Donn�es variables \\
\hline
opt   &    Suites applicatives additionnelles \\
\hline
srv   &    Donn�es pour les services \\
\hline
\end{tabular}
\end{center}
~\\

\section{Partitionnement}

\textbf{Conseils de partitionnement} \\
~\\
Pour une machine de type serveur, on appliquera un partitionnement\footnote{\url{http://www.tldp.org/HOWTO/Partition/}} plus r�fl�chi qu'un poste de travail (o� l'on se contente souvent d'isoler la partition contenant le r�pertoire home, la partition swap et le syst�me). Il n'existe pas de v�rit� absolue (cela varie selon l'utilit� de la machine et les habitudes des administrateurs) mais on passera en revue certaines recommandations. Voici quelques d�tails sur la taille des partitions destin�es � accueillir certains r�pertoires~: \\

\begin{itemize}
\item{/boot/ : partition d'environ 100 Mo}
\item{/ : partition sup�rieure � 100 Mo (conseil~: 500 Mo)}
\item{/tmp : quelques centaines de Mo ou davantage dans des cas particuliers}
\item{/var : partition sup�rieure � 250 Mo (conseil~: plusieurs Go si possible)}
\item{/usr : partition sup�rieure � 500 Mo (conseil~: quelques Go)}
\item{/home : � voir selon les quotas et le nombre d'utilisateurs}
\item{swap : partition sup�rieure � 16 Mo et inf�rieure � 2 Go (syst�mes 32-bits). On conseille souvent de mettre la m�me taille (ou le double) de la m�moire physique (sauf pour des applications sp�cifiques comme Oracle par exemple ou dans les cas de taille tr�s importante de la m�moire physique) et si possible proche du centre du disque.\\
D'autres espaces sp�cifiques notamment pour l'espace web, la base de donn�es pourront �tre cr��s selon les utilisations.}
~\\
\end{itemize}

\textbf{Exemple de partitionnement pour un disque d'environ 40 Go~:} \\
~\\
\begin{center}
\begin{tabular}{|c|c|c|}
\hline
Partition & Rep.montage & Taille  \\
\hline
sda1  &  /boot &  100 Mo  \\
\hline
sda3  &  /    &   500 Mo \\
\hline
sda4  &  /usr &   3 Go  \\
\hline
sda6  &  /var &   5 Go \\
\hline
sda7  &  /tmp  &  500 Mo \\
\hline
sda8  &  /mnt/ftp & 1 Go \\
\hline
sda9  &  swap  &  500 Mo \\
\hline
sda10 &  /home  &  20 Go \\
\hline
\end{tabular}
\end{center}
~\\
\textbf{Programmes de partitionnement} \\
~\\
{\emph{fdisk}}\\
Programme de partitionnement en ligne de commande tr�s r�pandu. Lire la page de manuel pour en savoir plus. \\
\textit{Lien :} \url{http://evolix.org/man/fdisk.html} \\
~\\
{\emph{cfdisk}}\\
Programme de partitionnement graphique en console. Il est utilis� par d�faut lors de l'installation sous Debian Woody. Son utilisation est plut�t simple et intuitive. \\
\textit{Lien :} \url{http://evolix.org/man/cfdisk.html} \\
~\\
{\emph{sfdisk}}\\
Programme de partitionnement en ligne de commande. sfdisk a quatre utilisations principales~: liste la taille d'une partition, liste les partitions d'un p�riph�rique, v�rifier une partition sur un p�riph�rique, et repartitionner un p�riph�rique.\\
\textit{Lien :} \url{http://evolix.org/man/sfdisk.html} \\
~\\
{\emph{parted}}\\
Programme de partitionnement en ligne de commande. GNU Parted est notamment indispensable pour g�rer des partitions sur des volumes d'une taille importante (sup�rieure � 2 To) en permettant la gestion des tables de partitions GPT\\
\textit{Lien :} \url{http://www.gnu.org/software/parted/} \\

\section{Gestion des disques}
~\\
%\emph{Disques de grande capacit�}\\
%\textit{Lien :} \url{http://www.freenix.org/unix/linux/HOWTO/Large-Disk-HOWTO.html}\\
%~\\
\emph{Syst�mes multi-disques}\\
\textit{Lien :} \url{http://tldp.org/HOWTO/Multi-Disk-HOWTO.html}\\
~\\
{\it{\bf LVM}}\\
~\\
LVM (Logical Volume Manager) permet une gestion plus ais�e que l'utilisation classique disques et partitions. Cela permet une plus grande flexibilit� pour redimensionner les volumes en fonction des besoins et des nouveaux disques disponibles.\\
\textit{Lien :} \url{http://www.tldp.org/HOWTO/LVM-HOWTO/} \\
~\\
{\it{\bf RAID}}\\
~\\
Le RAID (Redundant Arrays of Inexpensive Disks) est une solution bien connue pour obtenir des disques redondants afin de s�curiser les machines demandant une qualit� de service �lev�e. On parlera bien s�r ici de RAID logiciel.\\
\textit{Lien :} \url{http://www.tldp.org/HOWTO/Software-RAID-HOWTO.html} \\
~\\
{\small{Note : Nous recommandons d'utiliser le RAID logiciel et/ou LVM avec prudence pour les r�pertoires sensibles tels que /boot et le r�pertoire racine. Une solution de RAID mat�riel sera � privil�gier bien que plus co�teuse.}} \\
~\\
\textbf{Choix du syst�me de fichiers journalis� : ext3, ReiserFS, XFS ou JFS} \\
~\\
Il existe de nombreuses comparaisons entre ces syst�mes de fichiers mais il est difficile d'en tirer des conclusions g�n�rales car cela d�pend beaucoup de l'utilisation de la machine. La fiabilit� de ces syst�mes est d�sormais �prouv�e m�me si la prudence naturelle des administrateurs poussera � conserver ext3\footnote{\url{http://www.zip.com.au/~akpm/linux/ext3/}}, d�riv� du syst�me de fichier historique (Extended File System a �t� impl�ment� en 1992 et int�gr� � Linux 0.96c). En terme de performances, il semble se d�gager quelques constations comme la performance de ReiserFS\footnote{\url{http://www.namesys.com/}} pour les petits fichiers, un l�ger gain de XFS\footnote{\url{http://oss.sgi.com/projects/xfs/}} pour la copie de fichiers de grande taille mais une lenteur � l'effacement et la bonne tenue d'ext3 pour des op�rations classiques. JFS\footnote{\url{http://www-124.ibm.com/developerworks/oss/jfs/}} semblant l�g�rement moins performant pour le moment. \\
~\\
Sachant la perp�tuelle �volution des d�veloppements, les conditions dans lesquelles se d�roulent les tests et les autres param�tres (processeur, m�moire...) on se gardera bien de tirer des conclusions d�finitives. Disons simplement que l'utilisation de ext3 est encore assez raisonnable pour des serveurs classiques. En ce qui concerne ext4, il manque encore du recul pour l'utiliser dans des environnements critiques.\\
~\\
\textit{\textit{Liens :}}\\
\url{http://www.linux-france.org/article/sys/ext3fs/Benchmarks/} \\
\url{http://fsbench.netnation.com/} \\
~\\

\section{Packages}

Au fil des ann�es, les distributions Linux se sont vu dot�es de syst�mes de packaging multiples, et sont devenus des moyens tr�s utilis�s permettant d'installer des logiciels plus rapidement et plus facilement qu'avec une compilation classique :

\begin{itemize}
\item le RPM initialement d�velopp� par Redhat
\item les packages Slackware
\item les packages Debian (.deb)
\end{itemize}

Pour plus d'informations sur les RPMs, vous pouvez consulter \url{http://www.rpm.org/} et \url{http://www.lilit.be/formations/systeme_fichier/node5.html}.

\section{Configuration r�seau}

Au niveau des cartes r�seau, il est d�sormais assez rare de rencontrer des probl�mes de pilotes. Une fois les cartes r�seau correctement d�tect�es, on proc�dera � la configuration du r�seau. Si un serveur DHCP est pr�sent sur le r�seau, la configuration est automatique. On pr�f�rera n�anmoins une configuration statique pour des raisons de s�curit�.  Plus g�n�ralement la configuration r�seau se d�finit souvent dans un fichier sp�cifique selon les distributions. Voici un exemple d'un fichier \texttt{/etc/network/interfaces} pour Debian : \\
~\\
\texttt{auto eth0\\
iface eth0 inet dhcp\\
~\\
auto eth1\\
iface eth1 inet static\\
address 192.168.12.67\\
netmask 255.255.255.0\\
gateway 192.168.12.254}
~\\

En ce qui concerne la configuration DNS, c'est dans le fichier \texttt{/etc/resolv.conf} qu'on trouvera les adresses des serveurs de nom. On utilisera les outils \texttt{host} et \texttt{dig} pour s'assurer du bon fonctionnement. Voici un exemple de fichier~: \\
~\\
\texttt{search domain.tld \\
nameserver 192.168.12.71 \\
nameserver 62.4.17.69} \\
~\\
Les d�tails de la configuration r�seau et DNS ne sont pas abord�s ici, on se r�f�rera � de nombreuses documentations disponibles sur Internet ou dans les bonnes biblioth�ques. Passons tout de m�me en revue quelques outils indipensables � tout administrateur r�seau~: \\

~\\
\emph{ifconfig} permet de configurer les interfaces r�seau\\
Exemple : \texttt{ifconfig eth0 192.168.13.47}\\
 \texttt{ifconfig eth0:0 1.2.3.4 netmask 255.255.0.0}\\
~\\
\emph{route} gestion de la table de routage\\
Exemple : \texttt{route add -net 192.168.100.10/24 gw 192.168.100.1}\\
~\\
\emph{ip} gestion avanc�e de la configuration r�seau (routage, p�riph�riques, etc.) \\
Exemple : \texttt{ip addr add 10.0.0.1/24 dev eth0 label eth01} \\
~\\
\emph{netstat} informations avanc�es sur l'�tat r�seau \\
Exemple : \texttt{netstat -a -t} \\
~\\
\emph{traceroute} pour tester la route vers un h�te du r�seau\\
Exemple : \texttt{traceroute -v google.fr} \\
~\\
\textit{\textit{Lien :}} \url{http://perso-info.enst-bretagne.fr/~leroy/Unix/Guide/LinuxReseau.html}\\

\section{R�glages de base}

Revenons sur quelques r�glages de base~: \\
~\\
\emph{Boot loader} \\
~\\
Selon les distributions, le boot loader sera LILO ou Grub. Grub est plus flexible que LILO mais chacun poss�de des fonctionnalit�s diff�rentes qui peuvent s'av�rer utiles dans certains cas particuliers. Notons que sur un serveur en production, le choix du Boot loader n'est pas essentiel.\\
~\\
\emph{R�glage des locales} \\
~\\
Les locales sont l'environnement de localisation du syst�me. Cet environnement est utilis� par les programmes pour reconna�tre la langue et le jeu de caract�res que votre syst�me utilise. On peut choisir ses locales gr�ce au paquet locales. Actuellement, un choix s'offre entre le codage en ISO ou en UTF-8 (Unicode sur 8 bits). \\
Bien que de plus en plus d'applications soient compatibles avec le codage unicode, les utilisateurs h�sitent souvent � basculer en UTF-8. Pour un serveur, l'importance des locales est assez limit�e. Si l'on pourrait rester en ISO-8859-1 ou ISO-8859-15 (extension du codage europ�en avec le support de l'euro), il n'est d�sormais plus imprudent d'utiliser de l'UTF-8. \\
En ce qui concerne le choix de la langue, il peut �tre judicieux d'utiliser une locale de langue anglaise pour obtenir des messages d'erreur en anglais, ce qui facilite les recherches Internet par exemple. Ainsi, l'utilisation la plus raisonnable semble la locale en\_us.UTF-8 et l'ajout des locales fran�aises (fr\_FR et fr\_FR.UTF-8) est conseill�e (pour PHP ou PostgreSQL par exemple). \\
~\\

\emph{R�glage du clavier} \\
~\\
On pourra utiliser diff�rents claviers. On chargera la table de traduction en utilisant la commande loadkeys. Les tables disponibles sont contenues dans le r�pertoire /usr/share/keymaps\\
~\\
Exemples pour passer en azerty ou en qwerty sur certaines distributions~: \\
\texttt{\# loadkeys fr-latin0}\\
\texttt{\# loadkeys us-latin1}\\
~\\
Le programme kbdconfig peut aussi proposer une interface pour choisir son type de clavier. \\
~\\
\emph{Syst�me de messagerie}\\
~\\
Par d�faut, un syst�me du type Unix a besoin d'un syst�me de messagerie, ne serait-ce que pour envoyer des alertes syst�me � l'administrateur local. Si l'on n'a pas besoin de serveur de messagerie, on laissera donc un syst�me de mail avec une configuration locale uniquement. \\ 
~\\
\emph{Mise � l'heure}\\
~\\
Il est souvent essentiel pour un serveur d'�tre � l'heure. Pour cela le protocole NTP (Network Time Protocol)\footnote{\url{ftp://ftp.rfc-editor.org/in-notes/rfc2030.txt}} permet de se synchroniser sur un serveur de temps. Si l'on poss�de plusieurs serveurs, il est int�ressant d'avoir un serveur NTP\footnote{\url{http://www.ntp.org/}} sur lequel les autres machines locales vont se synchroniser. La synchronisation du serveur de temps ou du serveur isol� se fera sur plusieurs serveurs de temps officiels (certains sont en libre acc�s comme \texttt{ntp.tuxfamily.net} ou \texttt{swisstime.ee.ethz.ch}, sinon il faut demander un acc�s en envoyant un courrier �lectronique) \\
~\\
Concr�tement, on installe le paquet \textit{ntpdate} et on peut lancer une synchronisation dans un cron avec la commande suivante: \\
\begin{verbatim}
ntpdate ntp.evolix.net > /dev/null 2>&1 || echo \ 
"pas de mise a l'heure `date`" | /usr/bin/mail -s \
"[alert] ntpdate sur serveur.domain.tld `date`" admin@example.com
\end{verbatim}
\textit{Lien :}  \url{http://www.cru.fr/NTP/serveurs\_francais.html} \\


\section{Pr�sentation des principales distributions}

\subsection{Red Hat}
\url{http://www.redhat.com/}\\
\url{http://www.europe.redhat.com/documentation/rhl9/rhl-ig-x86-fr-9/}\\
\url{http://www.com.univ-mrs.fr/ssc/info/linux/install_linux.html}\\
~\\

Red Hat est la plus importante soci�t� avec une activit� d�di�e aux logiciels Open Source.

Fond�e en 1993, cette soci�t� bas�e aux �tats-Unis propose le d�ploiement d'infrastructures r�seau en se basant sur sa distribution : Red Hat Linux. Entr�e en bourse en 1999, Red Hat a chang� de strat�gie en 2003 en se consacrant uniquement au monde professionnel : Red Hat propose d�sormais une gamme de distributions payantes (Red Hat Entreprise) et se contente de sponsoriser un d�riv� de Red Hat : le projet Fedora. La soci�t� compte aujourd'hui plus de 700 salari�s r�partis dans plus de 20 pays.\\

Les versions :

\begin{itemize}
\item Redhat version 9, plus support�e, plus mise � jour
\item Redhat Entreprise Advanced Server
\item Redhat Entreprise Server
\item Redhat Entreprise Linux WorkStation
\end{itemize}

Depuis quelques mois, Redhat ne fournit plus directement de versions gratuites. Cette t�che incombe d�sormais � un groupe de d�veloppeurs annexes qui maintient la distribution Fedora. Fedora est une version gratuite de Redhat, qui permet �galement aux d�veloppeurs Redhat de "tester" certaines �volutions � grande �chelle avant que celles-ci soient int�gr�es.\\

L'installation de logiciels peut �tre r�alis�e de plusieurs fa�ons~:

\begin{itemize}
\item via les sources, en les compilant
\item via un RPM
\item via \textit{yum}, un syst�me qui ressemble fortement � apt pour les utilisateurs de Debian
\end{itemize}

Les mises � jour sont assur�es sur Fedora par \textit{yum}, ainsi que part \textit{uptodate} comme c'est le cas sur Redhat (utilisation texte ou graphique).\\

La s�curit� pour la distribution Redhat est g�r�e via la page \url{https://www.redhat.com/security/updates/}, o� toutes les mises � jour sont disponibles. Des RPMs sont disponibles.
~\\

\subsection{Mandriva}

\url{http://www.mandrakelinux.com/}\\
\url{http://www.mandrivalinux.com/fr/fdoc.php3}\\
\url{http://www.mandrivalinux.com/fr/ftp.php3}\\
\url{http://fr.wikipedia.org/wiki/Mandriva_Linux}\\
~\\

Mandriva est le produit d'une entreprise fran�aise, MandrakeSoft, cr��e en 1998. Gr�ce � une croissance rapide et de nombreux soutiens, la soci�t� MandrakeSoft compte plus de 120 salari�s et est cot�e en bourse. Elle impose sa distribution comme l'une des plus r�pandues dans le monde. Accessible dans plus de 50 langues, Mandriva est une distribution conviviale et accessible aux d�butants. Initialement appel�e Mandrake, elle f�t rebaptis�e Mandriva en 2005 (suite � des probl�mes judiciaires concernant le nom et au mariage de MandrakeSoft avec Conectiva, une distribution Linux venant d'Am�rique Latine).\\

Les versions actuelles~:

Il existe de tr�s nombreuses versions et 
Quelques principes : \\
Les versions "Free" ne contiennent que des logiciels libres, et sont t�l�chargeables gratuitement.\\
Les versions "Entreprise" sont, bien qu'Open Source, payantes et b�n�ficient d'un support sp�cial par Mandriva. La plupart permettent une inter-op�rabilit� entre les plates-formes Linux, Windows et Mac.\\

Quelques exemples :\\
\begin{itemize}
\item Mandriva Free : C'est tout un syst�me complet livr� avec tous les programmes qui peut �tre t�l�charg� gratuitement et librement distribuable, et grav� sur un DVD-ROM ou sur 3 CD-ROM.
\item Mandriva Powerpack : La version en bo�te (Powerpack) est payante et fournie avec des manuels imprim�s, pour aider les utilisateurs � d�couvrir le syst�me. Elle inclut �galement le droit � des services, comme de l'assistance technique.
\item Mandriva One : version "Live CD" utilisable sans rien installer
\item Mandriva Flash : version portable autobootable vendue sur une cl� USB
\item Multi Network Firewall (MNF) : version sp�ciale pour les machines de type routeur/firewall
\end{itemize}

Pour les probl�mes de s�curit�, Mandriva regroupe les informations sur \url{http://www.mandriva.com/security/}, et les utilisateurs b�n�ficient d'une grande r�activit�. Ainsi, une liste de diffusion et un flux RSS sont disponibles pour se tenir inform�.

\subsection{Gentoo}

\url{http://www.gentoo.org/}\\
\url{http://www.gentoo.org/doc/fr/handbook/index.xml}\\
\url{http://www.gentoo.org/doc/fr/gentoo-x86-quickinstall.xml}\\
~\\

Gentoo est une distribution "orient�e source". A l'image des syst�mes BSD dont elle d�rive, l'utilisateur compile son syst�me et ses logiciels. Cependant, il existe �galement des paquetages binaires disponibles, afin d'�viter cette phase de compilation.\\

Gentoo a �t� originairement con�u pour fonctionner sur architecture x86 uniquement. Mais, elle a �t� port�e sur de nombreuses autres architectures en raison de sa haute portabilit�. L'installation se fait � partir d'un CD Live et une documentation d'installation et d'utilisation tr�s compl�te est disponible. Notez qu'il existe un outil appel� Catalyst permettant de faire ses propres distributions bas�es sur Gentoo.\\

La distribution est tr�s orient�e compilation avec un syst�me de "flags" interne, qui permettent d'utiliser certaines caract�ristiques majeures, comme X (le serveur graphique), une version pr�cise du compilateur, ou encore le support de tel ou tel protocole dans tous les logiciels install�s. La compilation est omnipr�sente et des outils comme DistCC sont tr�s utilis�s afin d'en r�duire les co�ts. Cette distribution est bas�e sur le langage Python, utilis� dans la plupart des scripts inh�rents � la distribution (d�marrage, installation du r�seau etc.)\\

Vous pouvez prendre connaissance des failles de s�curit� et des �ventuels probl�mes apparaissant sur la distribution depuis \url{http://www.gentoo.org/security/en/index.xml}. La distribution est �galement ax�e sur la s�curit� avec un certain nombre de m�canismes int�gr�s par d�faut, et des ouvrages relatifs � la s�curit� qui ont �t� r�alis�s par l'�quipe (Gentoo Security Handbook). Les utilisateurs peuvent �galement prendre compte des changements oeuvr�s par mail (gentoo-announce@gentoo.org).

\subsection{Slackware}

\url{http://www.slackware.org/}\\
\url{http://www.trustonme.net/didactels/91.html}\\
~\\

Cette distribution historique et simple a �t� cr��e, et est toujours maintenue par Patrick Volkerding (depuis 1993). Elle ne repose que sur peu d'outils de configuration automatis�s, les outils traditionnels �tant privil�gi�s. Les deux objectifs principaux sont la simplicit� d'utilisation et la rapidit�.

Le syst�me de packages est g�r� gr�ce aux \textit{slacktools}, rempla�ants des \textit{pkgtools} de Slackware. Ils ont pu b�n�ficier constamment d'am�liorations, comme la gestion des d�pendances etc. Le syst�me de packaging RPM peut �galement �tre utilis� facilement afin d'utiliser des packages plus g�n�ralistes.\\

Contrairement � des distributions comme Debian, la compilation du noyau se fait de mani�re traditionnelle. Rappel~:

\begin{verbatim}
# make menuconfig
# make dep
# make bzImage
# make modules
# make modules_install
\end{verbatim}

Cependant, m�me si la distribution est relativement � jour, les personnes qui y contribuent sont peu nombreuses, et certaines am�liorations peuvent mettre du temps � �tre incorpor�es. Le projet Slackware a pour but de rester le plus Unix-like possible.\\

En cas de probl�me de s�curit�, une liste de diffusion d�di�e est accessible sur \url{http://www.slackware.org/security/}. Les mises � jour sont faites sous forme de packages � installer.

\subsection{Suse}

\url{http://www.novell.com/linux/suse/}\\
\url{http://www.novell.com/documentation/suse93/pdfdoc/user93-screen/user93-screen.pdf}\\

Suse est une soci�t� allemande qui appartient d�sormais au groupe Novell. La distribution du m�me nom est bas�e sur Slackware et le syst�me de paquets RPM de Redhat. La configuration peut �tre facilement r�alis�e par le Centre de Control YAST, qui est pass�, depuis plus d'un an sous licence GPL.

Vu l'orientation de Novell sur le monde des r�seaux, le rachat de Suse a marqu� l'int�gration de nombreux protocoles r�seau etc.\\

Novell propose actuellement plusieurs versions~:

\begin{itemize}
\item SUSE LINUX Professional : version professionnelle principalement mise en avant par Novell 
\item Novell Linux Desktop : version orient�e poste de travail
\item SUSE LINUX Enterprise Server : version sp�ciale pour les entreprises (certifi�e par exemple pour les machines IBM Xseries ou le logiciel Oracle)
\end{itemize}

Novell propose �galement de nombreux produits ou services bas�s sur Linux. On notera des sortes de packages comprenant une distribution Linux et des services~:

\begin{itemize}
\item Novell Linux Small Business Suite : bas�e sur Novell Linux Desktop
\item Novell Open Enterprise Server : bas�e sur SUSE LINUX Enterprise Server
\end{itemize}
~\\
~\\ 
Pour les probl�mes de s�curit�, Suse dispose d'un service en ligne \\
(� l'adresse \url {http://www.novell.com/linux/security/securitysupport.html}). ~\\
Une liste des failles ayant affect� Suse est disponible � la m�me adresse de fa�on textuelle, ou en flux RSS � l'adresse (\url{http://www.novell.com/linux/security/suse\_security.xml}).

\subsection{Debian}

Le projet Debian a �t� lanc� en ao�t 1993 par Ian Murdock. Debian est alors une nouvelle distribution produite de fa�on ouverte, dans l'esprit de Linux et de GNU. Debian a la r�putation d'�tre soigneusement et consciencieusement mise en place, maintenue et support�e. Cela a commenc� par la constitution d'un petit groupe tr�s soud� de hackers (codeurs) de logiciels libres. Graduellement, le groupe s'est agrandi pour devenir une vaste communaut� de d�veloppeurs et d'utilisateurs bien organis�e\footnote{\url{http://www.debian.org/intro/organization}}. Le projet Debian est bas� sur un contrat social\footnote{\url{http://www.debian.org/social\_contract.fr.html}},  des directives d�finissant les logiciels libres, DFSG - Debian Free Software Guidelines\footnote{\url{http://www.debian.org/social\_contract\#guidelines}} et une constitution\footnote{\url{http://www.debian.org/devel/constitution}} d�finissant l'organisation du projet. 
Organis�s de fa�on d�mocratique, le leader, le comit� technique et le secr�taire ainsi que chaque contributeur Debian poss�dent un r�le bien d�fini (responsable de paquets, de projets, etc.). Ainsi Debian n'est pas soumis � des exigences commerciales et se permet d'avoir un avenir serein. Aujourd'hui, on compte pr�s d'un millier de d�veloppeurs Debian r�partis dans le monde entier\footnote{\url{http://www.debian.org/devel/developers.loc}}, des dizaines de milliers de paquets\footnote{\url{http://www.debian.org/distrib/packages}}, le support de plusieurs 
architectures\footnote{\url{http://www.debian.org/ports/}} ainsi que des versions utilisant un noyau diff�rent de Linux (GNU/Hurd ou FreeBSD).

\textit{Lien~:} \url{http://www.debian.org/doc/manuals/project-history/} \\

Les versions~:\\

Debian propose trois ou quatre versions de distributions aux utilisateurs. \\
La distribution "stable" contient la derni�re distribution officiellement sortie de Debian. Il s'agit de la derni�re production de Debian qu'il est recommand� d'utiliser.\\
La distribution "testing" contient des paquets en attente d'entr�e dans la distribution "stable". Elle contient donc des versions assez r�centes des logiciels mais moins bien surveill�s par les d�veloppeurs Debian.\\
En pratique, la distribution "testing" passe par une p�riode de gel (freeze) o� son �volution est fig�e et seules les corrections de bogues ou mises-�-jours mineures sont accept�es (on parle parfois de distribution "frozen"). Apr�s de minutieuses v�rifications, la version "testing gel�e" peut remplacer la version stable. \\
\\
La troisi�me distribution est appel�e "unstable". C'est celle sur laquelle se concentre les activit�s de d�veloppement. Elle est utilis�e par "les d�veloppeurs et par ceux qui aiment vivre sur le fil" car c'est une sorte de laboratoire destin� � tester et corriger les bogues des paquets. Son nom Sid (nom de l'enfant qui casse les jouets dans \textit{Toys Story}) signifie officieusement : Still In Development.\\
\\
Les nouveaux paquets des d�veloppeurs arrivent en effet directement en unstable apr�s inspection et validation des ftpmasters (les personnes responsables des mises en ligne sur le serveur FTP principal). Chaque mise en ligne est r�percut�e sur les miroirs en moins de 24h (on peut retouver les nouveaux paquets des d�veloppeurs directement dans le r�pertoire Incoming\footnote{\url{http://incoming.debian.org/}}. Les paquets de la version "unstable" remplissant certains crit�res (absence de bogues critiques, dur�e sup�rieure � 10 jours dans "unstable", support de toutes les architectures et d�pendances toutes satisfaites) basculent dans la distribution "testing".\\
\\
Derni�re distribution plus particuli�re, il s'agit de la version "experimental" : il s'agit d'une archive mise � la disposition par des d�veloppeurs pour avoir des retours d'utilisateurs exp�riment�s sur des modifications importantes des logiciels (la version "experimental" n'est pas une distribution compl�te). Cette �tape est ind�pendante du processus de validation des paquets Debian.

On se rend donc compte des pr�cautions qui entourent le statut des paquets Debian. En contrepartie, ce processus prend du temps et les versions de paquets de la distribution "stable" ont souvent du retard par rapport aux derni�res versions des logiciels.

Voici le tableau des noms de distribution Debian (tir�s du Film Toy Story)~:

\begin{center}
\begin{tabular}{|c|c|c|c|}
\hline
-    & 0.01->1.1 & ao�t 1993->juin 1996 & - \\
\hline
Buzz & 1.1 & juin 1996 & Le ranger de l'espace \\
\hline
Rex  &   1.2 &    d�cembre 1996 &  le tyrannosaure \\
\hline
Bo   &   1.3 &    juillet 1997  &  La berg�re \\
\hline
Hamm  &  2.0 &    juillet 1998  &  Le cochon-tirelire \\
\hline
Slink &  2.1 &    mars 1999   &    Le chien � ressort \\
\hline
Potato & 2.2 &    ao�t 2000  &     Monsieur Patate \\
\hline
Woody  & 3.0 &    mai 2002  &      Le cow-boy \\
\hline
Sarge  & 3.1 &    juin 2005        &      Le chef des soldats \\
\hline
Etch   & 4.0  &   avril 2007      &      L'�cran magique \\
\hline
Lenny & 5.0 &    mars 2009	&      La paire de jumelles \\  
\hline
Squeeze & 6.0 & ? & L'extraterrestre � trois yeux\\
\hline
Sid    &  -  & toujours unstable &      L'enfant qui casse les jouets \\
\hline
\end{tabular} 
\end{center}

% http://kernel-handbook.alioth.debian.org/

% Copyright (c) 2004-2010 Evolix <info@evolix.fr>
%  Permission is granted to copy, distribute and/or modify this document
%  under the terms of the GNU Free Documentation License, Version 1.2
%  or any later version published by the Free Software Foundation;
%  with no Invariant Sections, no Front-Cover Texts, and no Back-Cover Texts.
%  A copy of the license is included at http://www.gcolpart.com/howto/fdl.html

\section{Focus sur Debian GNU/Linux}

\subsection{M�thodes d'installation}

Il existe donc diverses m�thodes d'installation\footnote{\url{http://www.debian.org/distrib/}}~: \\
~\\
\begin{itemize}
\item{{\emph{� partir de CD-ROM (ou DVD-ROM)\footnote{\url{http://www.debian.org/CD/}}}}}\\
Pour obtenir un jeu de CD-ROM (ou DVD-ROM) de Debian, on passera par l'un des revendeurs\footnote{\url{http://www.debian.org/CD/vendors/}} ou par le r�seau en t�l�chargeant des images\footnote{\url{http://www.debian.org/distrib/cd}}.\\
Les moyens de t�l�chargement sont sur des images compl�tes par HTTP/FTP sur des mirroirs\footnote{\url{http://www.debian.org/CD/http-ftp/}}, une construction d'image gr�ce � l'outil Jigdo\footnote{\url{http://atterer.net/jigdo/}} qui t�l�charge s�par�ment tous les fichiers du CD-ROM (ou DVD-ROM)\footnote{\url{http://www.debian.org/CD/jigdo-cd/}} ou encore gr�ce au Peer-to-peer comme le syst�me BitTorrent\footnote{\url{http://www.debian.org/CD/torrent-cd/}}.\\

\item{{\emph{� partir des disquettes\footnote{\url{http://www.debian.org/distrib/floppyinst}}}}}\\
On peut amorcer une installation de Debian � partir d'un nombre tr�s r�duit de disquettes 
(pratique notamment en cas d'absence de lecteur CD-ROM utilisable). 
Deux disquettes sont n�cessaires pour d�buter l'installation (disquettes Root et Rescue) 
puis quatre autres pour charger le noyau et ses modules.\\

\item{{\emph{� partir du r�seau}}}\\
Suite � l'installation � partir de disquettes ou � partir d'un CD-ROM minimal, on poursuit souvent l'installation par le r�seau pour installer le syst�me de base ainsi que les logiciels suppl�mentaires. En t�l�chargeant directement les paquets sur internet ou sur un mirroir local, il suffit donc d'une image minimale pour installer Debian GNU/Linux. N�anmoins, il est pr�f�rable de ne pas installer directement par internet car il faudrait ne pas connecter un serveur avant de l'avoir totalement s�curis� (et parce que les versions des paquets install�s par d�faut peuvent comporter des failles de s�curit�). On pr�f�rera des mirroirs locaux, voir par exemple apt-proxy\footnote{\url{http://apt-proxy.sourceforge.net/}} ou apt-move\footnote{\url{http://ptitlouis.dyndns.org/~ptitlouis/doc/}}.\\
~\\
On peut �galement amorcer compl�tement l'installation de Debian par le r�seau\footnote{\url{http://www.debianplanet.com/node.php?id=818}} si l'on poss�de un p�riph�riph�rique r�seau amor�able.\\
~\\
\end{itemize}

{\textit{Remarques sur les possibilit�s d'installation}}\\

Les possibilit�s d'installation de Debian sont nombreuses. On notera que le programme debootstrap\footnote{\url{http://www.debian.org/releases/stable/i386/ch-rescue-boot.fr.html\#s-dbootstrap-intro}} permet d'installer Debian � partir d'un syst�me de fichiers, qu'une installation automatis�e est possible en passant l'argument \textsl{preseed}\footnote{\url{http://d-i.alioth.debian.org/manual/fr.i386/ch04s07.html}} � l'amorce de l'installation par CD-ROM (ou DVD-ROM) et que la personnalisation de l'installation est possible.
\\
(voir \url{http://people.debian.org/~osamu/hackdi/}) \\

%TODO
%http://thierrylhomme.developpez.com/cfengine.html
%http://www.informatik.uni-koeln.de/fai/fai-guide-fr/
%utiliser Kickstart pour Ubuntu Hoary


\subsection{Installation et r�glages de base}

La version stable de Debian est toujours celle � privil�gier
pour de nouvelles installations. Dans certains cas, dans les mois
pr�c�dents la sortie de la future version stable, celle-ci peut
�tre install� si le serveur n'est pas trop critique. Dans ces
p�riodes charni�res, il faut bien prendre en compte qu'il existera
toujours des administrateurs qui affirmeront le contraire. Il faut donc
essayer de se faire un avis objectif � partir de plusieurs serveurs de
test avant toute d�cision importante.\\
~\\
%http://lists.debian.org/debian-devel-announce/2004/08/msg00001.html
%http://ftp-master.debian.org/testing/hints/freeze
%http://bugs.debian.org/release-critical/

Voici les recommandations que nous faisons � nos clients~:\\
~\\
\textit{Pour un serveur en production}\\
~\\
Nous conseillons d'utiliser Debian stable, � moins de vraiment n�cessiter beaucoup de fonctionnalit�s pr�sentes dans la prochaine version stable, et dans ce cas la version Debian testing peut �tre envisag�e. \\
~\\
\textit{Pour un serveur en semi-production}\\
~\\
Dans un environnement non critique et dans l'optique est de pr�parer une future mise en production, Debian testing peut �tre utilis�e si la version stable date de plus de 12 � 18 mois ; sinon l'utilisation de Debian stable reste � privil�gier. \\
~\\
\textit{Pour un poste de travail}\\
~\\
Nous conseillons l'utilisation de Debian testing (voire unstable) afin d'avoir des versions r�centes (mais n�anmoins test�es) des logiciels de bureautique. Cela permet de passer de fa�on souple vers la prochaine version stable lors de sa sortie et de l'utiliser pendant quelques mois avant de rebasculer vers Debian testing. \\
~\\
\textbf{Installation de Debian}\\
~\\
Revenons bri�vement sur l'installation de Debian g�r�e par le Debian-Installeur\footnote{\url{http://www.debian.org/devel/debian-installer/}}~: \\
\begin{itemize}
\item{D�marrage, d�tection des cartes r�seau, des paquets udebs (.udeb est un format particulier de paquet pour l'installeur)}
\item{Chargement �ventuel de pilotes ou firmware}
\item{Outils de partitionnement, installation de base}
\item{D�tection avanc�e des p�riph�riques gr�ce � discover\footnote{\url{http://d-i.alioth.debian.org/manual/fr.i386/index.html}} ce qui permet de charger automatiquement les modules ad�quats}
\item{Installation de GRUB, red�marrage}
\item{Configuration de base}
\end{itemize}
~\\

Apr�s cette installation, on proc�dera � une mise-�-jour g�n�rale (sur un miroir local si possible), une installation d'outils pratiques, une compilation du noyau puis une s�curisation. Voici quelques outils pratiques � installer~: ssh vim less mailx metche sudo munin log2mail apt-listchanges apticron evocheck\\

\subsection{Syst�me de packages Debian}

L'installation de nouveaux logiciels Open Source sous environnement Linux peut toujours se faire selon la m�thode classique de recompilation des sources. N�anmoins cette m�thode est peu ais�e pour la gestion des d�pendances, des mises-�-jour, etc. Ainsi les distributions utilisent souvent des syst�mes de paquetage. Debian utilise des paquets � l'extension \texttt{.deb} qui fournissent non seulement les binaires pr�compil�s (le plus souvent) mais �galement des m�thodes de gestion pour faciliter la manipulation de ces paquets. \\


L'utilitaire basique de manipulation des paquets Debian porte le nom de \texttt{dpkg}. Il est important de bien ma�triser les diff�rentes options de \texttt{dpkg} :
\begin{verbatim}
dpkg --unpack : 
\end{verbatim}
le paquet est d�paquet� mais n'est pas configur�
\begin{verbatim}
dpkg --configure
\end{verbatim}
configuration d'un paquet
\begin{verbatim}
dpkg -i
\end{verbatim}
installation compl�te
\begin{verbatim}
dpkg -r 
\end{verbatim}
supression du paquet
\begin{verbatim}
dpkg -r -P
\end{verbatim}
suppression compl�te (fichiers de configuration compris)
\begin{verbatim}
dpkg -L 
\end{verbatim}
affiche la liste des fichiers appartenant au paquet
\begin{verbatim}
dpkg -S 
\end{verbatim}
recherche un fichier dans les paquets install�s
\begin{verbatim}
dpkg -l
\end{verbatim}
liste tous les paquets install�s sur le syst�me.
\begin{verbatim}
dpkg --get-selections / --set-selections
\end{verbatim}
Donne/installe la/une liste des paquets install�s
~\\
Debian poss�de un programme de gestion avanc�e de paquets appel� APT (Advanced Packaging Tool).\\
Ce syst�me g�re les paquets d'apr�s une liste (Packages.gz) et permet de g�rer les d�pendances, les mises-�-jour particuli�res ou globales ou encore les conflits. Le fichier qui r�f�rence toutes les sources de paquets disponibles est \texttt{/etc/apt/sources.list}. On peut g�rer le contenu de ce fichier � l'aide de la fonction \texttt{apt-setup}. \\
~\\
Exemple de fichier \texttt{/etc/apt/sources.list} pour Debian Etch :\\
\begin{verbatim}
deb http://security.debian.org/ etch/updates main contrib non-free
deb ftp://ftp2.fr.debian.org/debian etch main contrib non-free
deb-src ftp://ftp2.fr.debian.org/debian etch main contrib non-free
\end{verbatim}
~\\
\texttt{dselect} est l'outil de gestion des paquets historique (dpkg a une d�pendance envers dselect~!). Son utilisation n'est pas forc�ment ais�e pour les novices. \\
~\\
Lien : \url{http://www.debian.org/releases/woody/i386/dselect-beginner}\\
~\\
La commande \texttt{apt-get} est une interface pour APT. Voici quelques commandes souvent utilis�es~: \\
~\\
\texttt{apt-get update}\\
Cette commande resynchronise les informations sur les fichiers disponibles � partir des endroits sp�cifi�s dans le \texttt{sources.list}. Cette commande r�cup�re donc les fichiers Packages.gz et les analyse de mani�re � rendre disponibles les informations concernant les nouveaux paquets.\\
~\\
\texttt{apt-get upgrade}\\
Cette commande met � jour tous les paquets dont une version plus r�cente est disponible sans supprimer de paquets, ni en ajouter. \\
~\\
\texttt{apt-get dist-upgrade}\\
Cette commande met � jour les paquets (comme upgrade) en utilisant une gestion intelligente des changements de d�pendances. Ainsi des paquets pourront �tre supprim�s et des nouveaux paquets pourront �tre install�s.\\
~\\
\texttt{apt-get install <paquet>}\\
Cette commande installe le paquet <paquet> ainsi que toutes ses d�pendances n�cessaires.
\begin{verbatim}
apt-get remove --purge <paquet>
\end{verbatim}
Cette commande supprime le paquet <paquet> ainsi que tous ceux qui en d�pendent.\\
~\\
\texttt{apt-get clean}\\ 
Cette commande supprime les paquets install�s du r�pertoire de cache d'APT (lib�re de l'espace disque). On peut �galement utiliser autoclean qui va supprimer uniquement les paquets les plus anciens et inutiles.\\
~\\

D'autres interfaces utilisent APT~:\\
~\\
\begin{itemize}
\item[{\textbf{aptitude}}]
Outil de gestion des paquets directement au-dessus d'APT. Il peut �tre utilis� comme alternative � apt-get (les "commandes apt-get" sont accept�es) et offre certains avantages suppl�mentaires\footnote{\url{http://www.debian.org/doc/manuals/reference/ch-package.fr.html\#s-aptitude}} (par exemple, aptitude retient les d�pendances install�es et les supprime lorsqu'elles ne sont plus n�cessaires). Le Debian-Installeur utilise d'ailleurs aptitude et l'utilisation de cet outil est d�sormais recommand�e. \\
~\\
\item[{\textbf{synaptic}}]
Outil graphique de gestion des paquets en GTK+ bas� sur APT. Il permet d'utiliser la plupart des fonctions (recherche, installation, mise-�-jour, etc.)\\
\end{itemize}
~\\
Enfin de nombreux outils suppl�mentaires existent pour g�rer les paquets~:
~\\
\begin{itemize}
\item[{\textbf{apt-cache}}] utilitaire permettant d'obtenir un certain nombre d'informations sur les paquets et le cache d'APT.\\
\item[{\textbf{apt-file}}] utilitaire permettant d'effectuer des recherches dans le syst�me de paquets d'APT. � la diff�rence d'apt-cache, on peut rechercher des informations pr�cises (noms des fichiers du paquet) sur l'ensemble des paquets m�me si ils ne sont pas install�s.\\
Exemple : \texttt{apt-file search stdio.h}\\
\item[{\textbf{apt-listbugs}}] outil qui liste automatiquement les bogues critiques avant d'installer les nouveaux packages, et, si des bogues critiques sont r�f�renc�s, qui vous demande de confirmer ou non la mise � jour. Il permet de "blacklister" les paquets ayant des bogues critiques et donc de ne pas les installer.\\
\item[{\textbf{apt-show-versions}}] liste des paquetages install�s.\\
\item[{\textbf{apticron}}] script pour signaler les mises-�-jour disponibles.\\
\end{itemize}
~\\
Lien : \url{http://www.debian.org/doc/manuals/apt-howto/}\\
~\\

\newpage

\emph{Probl�mes dans la gestion des paquets}\\
~\\
Bien conna�tre le syst�me de gestion de paquets Debian est une condition n�cessaire pour un administrateur. Ainsi, s'il arrive un probl�me lors de l'installation ou la mise-�-jour (assez rare avec la version stable), on pourra d�boguer les conflits. On gardera � l'esprit quelques commandes essentielles � utiliser avec pr�caution :\\ 
\begin{verbatim}
# apt-get -f install
\end{verbatim}
commande qui tente de corriger les �ventuels probl�mes lors d'une installation ou mise-�-jour
\begin{verbatim}
# dpkg --ignore-depends --force-all
\end{verbatim}
commande qui permet de forcer l'installation d'un paquet
\begin{verbatim}
# dpkg --configure -a
\end{verbatim}
commande qui relance les �tapes de configuration de tous les paquets pr�sents sur le syst�me

% http://www.cyberdogtech.com/firewalls/

%%%%%%%%%%%%%%%%%%%%%%%%%%%%%%%%%%%%%%%%%%%%%%%%%%%
% Copyright (c) 2005 eVoLiX. Tous droits reserves.%
%%%%%%%%%%%%%%%%%%%%%%%%%%%%%%%%%%%%%%%%%%%%%%%%%%%

\section{Le noyau Linux}

\subsection{Pr�sentation}
~\\
Andrew Tanenbaum, professeur, d�veloppa en 1985 Minix, un syst�me d'exploitation minimal inspir� du syst�me UNIX Time-Sharing System version 7. Destin� � enseigner le concept des syst�mes d'exploitations, Minix fut la source d'inspiration de Linus Torvalds, un �tudiant finlandais, qui d�veloppa un nouveau syst�me d'exploitation baptis� Linux. Compl�tement r�-�crit (principalement en langage C), Linux fut tout d'abord d�velopp� pour les ordinateurs de type i386. Linus Torvalds choisit la licence GPL pour son d�veloppement et rapidement, gr�ce � Internet, il re�ut l'aide de plusieurs informaticiens. L'explosion des r�seaux et d'Internet a largement favoris� le d�veloppement collaboratif de Linux, au point que certains le consid�rent comme le premier produit d'Internet. Linux est g�n�ralement compil� avec GCC et plus accompagn� d'outils provenant du projet GNU afin de fournir un syst�me d'exploitation utilisable. C'est pourquoi on a tendance � qualifier ce syst�me d'exploitation de "GNU/Linux" \footnote{\url{http://www.gnu.org/gnu/why-gnu-linux.fr.html}} \\
La premi�re version de Linux sortit en 1991, la version 1.0 sortit en 1993 et la version 2.0 sortit en 1996. Aujourd'hui, Linux supporte plusieurs architectures (Alpha, MIPS, SPARC, PPC, ...) et compte plusieurs millions de lignes de code. On peut t�l�charger Linux sur le site \url{http://www.kernel.org}. Sa branche stable actuelle est la 2.6.x \\

~\\
 x.y.z \\
~\\
x d�signe la branche (num�ro de version majeure), y compl�te le num�ro de version et z est le num�ro de release (c'est-�-dire le nombre de publications de cette version). Si y est pair, il s'agit d'une version stable, si y est impair il s'agit d'une version en cours de d�veloppement. \\
 Depuis la version 2.6.11, la num�rotation a un peu chang� avec l'introduction d'un quatri�me chiffre indiquant des changements mineurs de versions (quelques patches peu cons�quents). \\

Des suffixes tels que preN ou rcN (pr�patches), bkN (snapshots), acN (patches d'Alan Cox) ou mmN (patches d'Andrew Morton) indiquent des versions particuli�res du noyau. \\

On peut conna�tre la version actuelle gr�ce � la commande \texttt{uname} qui affiche  des  informations  concernant  la machine et le syst�me d'exploitation sur lequel il est invoqu�. \\

~\\
Exemple : \\
\texttt{
\$ uname -r \\
2.6.10-rc2laptop221104} \\
~\\



%Multi-utilisateurs \\
%Principes utilisateurs/groupes \\
%Processus \\
%noyau monolithique != micro noyau \\
%approche modulaire \\
%syst�me de fichiers (d�finitions POSIX d'un fichier) \\
%droits d'acc�s \\
%mode User/mode Kernel \\
~\\
Liens:\\
\url{http://www.kernelnewbies.org/}\\
\url{http://www.bertolinux.com/}\\

\subsection{Compilation}
~\\
Sous Debian, le noyau compil� se trouve dans le r�pertoire /boot sous le nom vmlinuz-x.y.z
Les sources du noyau se trouvent dans le r�pertoire /usr/src/linux-x.y.z (ou kernel-source-x.y.z).
Historiquement on retrouve un lien /usr/src/linux pointant vers le r�pertoire des sources du noyau.
On trouve �galement les modules dans /lib/modules/x.y.z, les ent�tes dans /usr/src et les tables de symboles\footnote{\url{http://www.dirac.org/linux/system.map/}} du noyau dans system.map-x.y.z (souvent dans le r�pertoire /boot )
~\\
~\\
Une fois l'installation de Linux termin�e, on proc�dera �ventuellement (surtout sur un serveur) � la compilation d'un noyau adapt� aux besoins de la machine. On privil�giera un noyau d�barrass� des modules inutiles~: toutes les options strictement n�cessaires devront �tre compil�es en dur (sauf exception).\\

On va donc r�cup�rer les sources du noyau Linux. la proc�dure la plus classique est de prendre le noyau sur le site officiel \url{http://www.kernel.org/}. On prendra garde � bien v�rifier l'int�grit� des sources t�l�charg�es~: \\ 
~\\
\begin{verbatim}
$ wget http://kernel.org/pub/linux/kernel/v2.6/linux-2.6.z.tar.bz2.sign
$ gpg --keyserver wwwkeys.pgp.net --recv-keys 0x517D0F0E
$ gpg --verify linux-2.6.z.tar.bz2.sign linux-2.6.z.tar.bz2
\end{verbatim}
~\\
La recompilation dite classique d'un noyau consiste � v�rifier les d�pendances (\texttt{dep}), nettoyer les sources (\texttt{clean}), compiler le noyau en lui-m�me (\texttt{bzImage}), puis les modules (\texttt{modules}) et les installer (\texttt{modules\_install}) :\\
~\\
\begin{itemize}
\item[\texttt{make dep}] v�rifie les d�pendances\\
\item[\texttt{make clean}] fait un peu le m�nage\\
\item[\texttt{make bzImage}] compile le noyau\\
\item[\texttt{make modules}] compile les modules\\
\end{itemize}
~\\
\textit{makes modules\_install}
~\\
\textit{cp arch/i386/boot/bzImage /boot/vmlinuz-new}\\


Avec Debian, il existe les sources Debian du noyau Linux. Il s'agit des sources originales patch�es par Debian. On t�l�chargera ces sources gr�ce � APT~: \\
~\\
\texttt{apt-cache search kernel-source*} \\
~\\
Lorsque l'on veut patcher les sources du noyau, il faudra �viter d'appliquer des patches incompatibles. C'est pourquoi il peut �tre pr�f�rable de patcher � partir des sources originales. Il existe de nombreux patches pour le noyau Linux. Pour un serveur, on s'int�ressera au patches de s�curit� grsecurity\footnote{\url{http://www.grsecurity.net/}} proposant un certain nombre de fonctionnalit�s. Voici la proc�dure pour l'installer~: \\
~\\
%http://www.cgsecurity.org/Articles/2-MISC/Protections-2/
~\\
\texttt{\$ tar -jxvf linux-2.6.z.tar.bz2}\\
\texttt{\$ patch -p0 < grsecurity-2.0-2.6.z.patch}\\
~\\
La phase la plus d�licate est en r�alit� le choix des options du noyau. En effet, il faut lire attentivement les explications de chaque option pour d�terminer si on doit l'activer pour notre machine. Pour lancer le choix des options du noyau, on fera:\\
~\\
\texttt{cd linux-2.6.z.tar.bz2}\\
\texttt{make menuconfig}\\
~\\
On pourra �galement utiliser \texttt{make config} mais son utilisation est moins ais�e.
Le choix des options est stock� dans le fichier \texttt{.config}.
En cas de changement de sources (par exemple en cas de changement de version du noyau), on peut r�utiliser ce fichier \texttt{.config} en ne choisissant que les nouvelles options. Pour cela on le transf�rera � la racine des nouvelles sources et on lancera la commande~:\\
~\\
\texttt{make oldconfig}\\
~\\
~\\
Debian propose des outils pour compiler simplement. La commande \texttt{make-kpkg} permet de construire des paquets Debian contenant un noyau compil� pr�t � �tre install�. On proc�dera ainsi :\\
~\\
\texttt{
make-kpkg clean\\
make-kpkg kernel\_image kernel\_headers kernel\_source kernel\_doc\\
dpkg -i ../kernel-*-2.6.z.xxx.deb\\
}
~\\
Les options de \texttt{make-kpkg} permettent de sp�cifier un nom particulier pour le noyau (-~-append-to-version), de compiler avec initrd (-~-initrd), de compiler avec des modules (-~-added-modules), etc.
~\\
~\\


%%%%%%%%%%%%%%%%%%%%%%%%%%%%%%%%%%%%%%%%%%%%%%%%%%%
% Copyright (c) 2005-2011 eVoLiX. Tous droits reserves.%
%%%%%%%%%%%%%%%%%%%%%%%%%%%%%%%%%%%%%%%%%%%%%%%%%%%

\chapter{Administration Syst�me et R�seau}

\section{Gestion des droits}

Sous les syst�mes de type Unix ou Linux, il existe plusieurs types de fichiers : les fichiers, les r�pertoires, les liens symboliques, les fichiers-p�riph�riques.\\
~\\
Un fichier appartient � un utilisateur (en fait un num�ro d'utilisateur) et � un groupe (en fait un num�ro de groupe).\\
On peut changer l'utilisateur par la commande:
\begin{verbatim}
$ chown <new_user> fichier
\end{verbatim}
~\\
On peut changer le groupe par la commande:
\begin{verbatim}
$ chgrp <new_group> fichier
\end{verbatim}
~\\
On peut changer l'utilisateur et le groupe par la commande:
\begin{verbatim}
$ chown <new_user>.<new_group> fichier
\end{verbatim}
~\\
Les 3 droits fondamentaux sont la lecture, l'�criture et l'�xecution.\\
Pour un fichier, ces 3 droits sont d�finis pour 3 cat�gories: l'utilisateur, le groupe et le "reste du monde". Pour chacune de ces cat�gories: on note les droits sous la forme rwx.\\
\textbf{r} correspond � l'�criture, \textbf{w} � l'�criture et \textbf{x} � l'ex�cution. Par exemple, \textbf{r-x} correspond � des droits de lecture et d'ex�cution autoris�s, mais l'�criture interdite.\\
Un notation pratique est de voir \textbf{rwx} comme 3 bits. L'autorisation correspond � 1 et l'interdication � 0.\\
Par exemple \textbf{r-x} correspond � 101. En base 10, 101=1*4+0*2+1*1=5. Cela correspond donc � 5.On notera c�te � c�te les droits des 3 cat�gories.\\
Par exemple \textbf{rwxr-xr--} correspond � des droits \textbf{rwx} pour l'utilisateur, \textbf{r-x} pour le groupe et \textbf{r--} pour le reste du monde. Si vous avez bien suivi, \textbf{rwxr-xr--}=111101100=754\\
On ajoute (ou retire) des droits avec la commande :
\begin{verbatim}
$ chmod <cat�gorie>+<nouveau_droit> fichier
\end{verbatim}\\
~\\
\texttt{<cat�gorie>} peut �tre u (utilisateur), g (groupe) ou a (all=reste du monde). \texttt{<nouveau\_droit>} peut �tre r,w ou x\\
Pour retirer, on mettra un - � la place du +\\
On peut changer compl�tement les droits d'un fichier par la commande:
\begin{verbatim}
$ chmod <nouveaux_droits> fichier
\end{verbatim}
Par exemple, \texttt{<nouveaux\_droits>}=777 pour autoriser tout par tout le monde.\\
~\\
Pour un r�pertoire, on pourra lister les fichiers dans ce r�pertoire si on a les droits \textbf{r} du r�pertoire (et ceux des r�pertoires sup�rieurs). Si on a pas ces droits, on ne pourra pas lister. Le droit \textbf{rx} permet en plus d'entrer dans ce r�pertoire. Sans lui, on ne pourra pas effectuer des op�rations sur un des fichiers contenus dans ce r�pertoire. Par exemple, le r�pertoire \texttt{/root/} n'a souvent aucun droit pour "all" et donc on ne peut rien faire dans ce r�pertoire si l'on est pas root.\\
~\\
Notons que le droit \textbf{x} seul permet de traverser le r�pertoire mais pas d'y entrer ou de le lister\\
~\\
Attention, pour pouvoir effacer un fichier ou un r�pertoire vide, il suffit d'avoir les droits \textbf{wx} sur le r�pertoire contenant ! (avoir uniquement de droit \textbf{w} ne sert -a priori- � rien pour un r�pertoire). Notez que dans un r�pertoire avec droit \textbf{wx}, on peut effacer les fichiers (ou r�pertoires vides) sur lesquels on n'a pas les droits (par contre, on ne peut pas effacer les r�pertoires non vides sur lesquels on a pas les droits \textbf{wx}).\\
~\\
En plus de cela, il existe des droits sp�ciaux: setuid, setgid et sticky bit\\
- Un fichier ex�cutable peut �tre \texttt{setuid}, c'est-�-dire qu'au lieu d'�tre ex�cut� avec les droits de l'utilisateur qui le lance, il sera ex�cut� avec les droits du propri�taire de l'ex�cutable. Ceci s'av�re assez dangereux notamment pour les ex�cutables \texttt{setuid} root. L'exemple-type est le programme \texttt{passwd} qui permet de changer de mot de passe. Il est ex�cutable par un utilisateur mais il est \texttt{setuid} root car seul root peut �crire dans les fichiers \texttt{/etc/passwd} et \texttt{/etc/shadow}\\
- Un fichier ex�cutable peut �tre \texttt{setguid}. Il s'agit de la m�me notion que celle vue ci-dessus pour le groupe. Le fichier est donc ex�cut� avec les droits du groupe auquel il appartient.\\
- Un fichier ex�cutable peut �tre \texttt{sticky}, c'est-�-dire avoir le \texttt{sticky bit} positionn�. Cela signifie qu'il reste en m�moire m�me apr�s la fin de son ex�cution afin d'�tre relanc� plus rapidement. Attention, seul root peut positionner le \texttt{sticky bit}.
%� pr�ciser
- Un r�pertoire peut �tre \texttt{setgid}. Cela signifie que tous les fichiers cr��s dans ce r�pertoire appartiendront au m�me groupe que le r�pertoire.\\
- Un r�pertoire peut �tre \texttt{sticky} bit. Cela signifie dans ce r�pertoire, un utilisateur ne pourra effacer que les fichiers qui lui appartiennent. L'exemple-type est le r�pertoire /tmp o� tout le monde peut �crire mais o� l'on ne peut effacer que ce que l'on a cr��.\\
Ces droits sp�ciaux sont not�s \texttt{sst} o� le premier \texttt{s} correspond au \texttt{setuid}, le second au \texttt{setgid} et le \texttt{t} au \texttt{sticky bit}.\\
On �crira �galement cela sous la forme de bits. Par exemple s-t=101=5 On ajoute (ou retire) des droits sp�ciaux avec la commande:
\begin{verbatim}
$ chmod +<droit_sp�cial> fichier
\end{verbatim}
~\\
\texttt{<droit\_sp�cial>} peut �tre \texttt{s (setuid+setgid)} ou \texttt{t (sticky bit)}\\
On change compl�tement les droits sp�ciaux et les droits d'un fichier par la commande:
\begin{verbatim}
$ chmod <nouveaux_droits_sp�ciaux><nouveaux_droits> fichier
\end{verbatim}
~\\
\texttt{<nouveaux\_droits\_sp�ciaux>} s'�crit en base 10. Par exemple, 5 pour setuid et \texttt{sticky bit}.\\
~\\
Il faut aussi d�finir la politique de gestion des droits de la machine, c'est-�-dire se poser la question "Qui a le droit de faire quoi~?" \\
D�finissons tout d'abord les droits au niveau des donn�es utilisateurs. Les droits par d�faut sont g�r�s par la directive \texttt{umask}. On mettra donc, selon sa politique, dans le fichier \textit{/etc/profile} (et dans login.defs pour g�rer lors de la cr�ation de l'utilisateur) :\\
~\\
\textit{umask 022} : pour que les donn�es utilisateurs soient visibles par tous les utilisateurs\\
~\\
\textit{umask 027} : pour que les donn�es d'un utilisateur soient visibles entre eux par les utilisateurs du m�me groupe\\
~\\
\textit{umask 077} :  pour que seul l'utilisateur puisse lire ses fichiers par d�faut\\
~\\
Il faut ensuite g�rer les droits sur les p�riph�riques. Les diff�rents p�riph�riques appartiennent souvent � un groupe (cdrom, audio, video, etc.) et on peut g�rer les droits en ajoutant ou non les utilisateurs � ces groupes. On ne d�taillera pas trop cette proc�dure car elle est surtout valable dans le cas o� il s'agit de postes de travail accessible physiquement aux utilisateurs.\\
~\\
Il faut �galement penser � mettre des protections sur les r�pertoires et partitions accessibles � l'utilisateur (on traitera le cas des journaux syst�mes � part : voir par la suite). Ces protections ne sont pas infaillibles mais constituent un premier rempart dans le cas d'attaques d'un utilisateur inexp�riment� (souvent appel�s script-kiddy). Voici un exemple de protection que l'on peut mettre dans le fichier \textit{/etc/fstab}~:\\
~\\

\begin{verbatim}
/dev/sda8 /tmp ext3 defaults,nodev,nosuid,noexec,usrquota,grpquota 0 2
# disques amovibles
/dev/fd0 /mnt/fd0 ext2 defaults,users,nodev,nosuid,noexec 0 0
/dev/fd0 /mnt/floppy vfat defaults,users,nodev,nosuid,noexec 0 0
/dev/hdc /mnt/cdrom iso9660 ro,users,nodev,nosuid,noexec 0 0
\end{verbatim}

Remarque~: certains paquets (screen, PostgreSQL, etc.) n�cessitent d'ex�cuter un script dans \textit{/tmp/}. On fera donc~:
\begin{verbatim}
# mount -o exec,remount /tmp
\end{verbatim}

Avant d'installer des paquets, puis on refera~:
\begin{verbatim}
# mount -o remount /tmp
\end{verbatim}

Plus g�n�ralement on peut v�rifier ce que peut voir un utilisateur par les commandes~:

\begin{verbatim}
$ find / -type f -a -perm +006 2>/dev/null}
$ find / -type d -a -perm +007 2>/dev/null}
\end{verbatim}

On prendra garde par exemple aux donn�es sensibles des fichiers de configuration ou des scripts (notamment des donn�es web). On peut par exemple voir les fichiers du r�pertoire /etc non visibles par un utilisateur~:

\begin{verbatim}
# find /etc -type f -a -perm 600 -a -uid 0
\end{verbatim}

On peut �galement restreindre les droits sur certaines applications dangereuses.\\
Exemple :\\
\begin{verbatim}
chmod o-x /usr/bin/nmap
chmod -s /bin/ping
\end{verbatim}
On pourra autoriser l'utilisation de n'importe quel programme (m�me des programmes administrateurs) gr�ce au logiciel sudo\footnote{\url{http://www.courtesan.com/sudo/}}. Son fichier de configuration est \textbf{/etc/sudoers}. Voici un exemple~: \\
~\\
\texttt{User\_Alias STAFF=jo,zette\\
Cmnd\_Alias NET=/bin/ping,/usr/bin/traceroute,/usr/bin/nmap\\
~\\
root ALL=(ALL) ALL\\
STAFF ALL=(ALL) NET\\
}\\
~\\
\textit{ \# /etc/init.d/sudo restart}
~\\

\section{Quotas}

Il est souvent int�ressant de d�finir des quotas~:\\
~\\
\emph{Configuration du noyau :} CONFIG\_QUOTA \\
~\\
Ajoutez les options \textit{usrquota} et \textit{grpquota} pour les partitions concern�es dans le fichier \textit{/etc/fstab}~:\\
{\small \texttt{/dev/hdc8 /home ext3 rw,nosuid,nodev,exec,nouser,auto,async,usrquota,grpquota 0 2}}\\
~\\
\texttt{\# touch /home/aquota.user /home/aquota.group}\\
\texttt{\# chmod 600 /home/aquota.* \&\& chown root:root /home/aquota.*}\\
\texttt{\# mount -v -o remount /home}\\
\texttt{\# apt-get install quota quotatool -> warnquota}\\
\texttt{\# update-rc.d -f quotarpc remove}\\
\texttt{\# quotacheck -auvg}\\
\texttt{\# quotaon -auvg}\\
~\\
On peut maintenant cr�er un utilisateur qui va servir d'exemple pour les quotas des autres utilisateurs~:\\
\begin{verbatim}
# adduser --home /dev/null --shell /bin/false --ingroup nogroup --disabled-password forquota
# edquota -u forquota
\end{verbatim}
~\\
La taille d'un "block" est habituellement de 1024 octets sous Linux. Elle peut varier selon les options des syst�mes de fichiers.\\
~\\
%\begin{verbatim}
%dumpe2fs /dev/myvol1-home |grep "Block size"
%\end{verbatim}

Les inodes repr�sentent le nombre maximum de fichiers ou de r�pertoires que l'on pourra cr�er.\\
%http://lea-linux.org/admin/admin_fs/quotas.html
%http://www.enel.ucalgary.ca/People/Norman/enel315_winter1997/disk_quotas/

~\\
On pourra �galement d�finir le d�lai (grace period) qui d�finit le temps au-del� duquel la limie douce devient limite dure. On change ce d�lai avec la commande :\\
\texttt{\# edquota -t}\\
~\\
On peut ensuite appliquer les quotas � chaque utilisateur par la commande :\\
\texttt{\# edquota -p forquota USER}\\
~\\
On peut utiliser le fichier adduser.conf pour imposer un quota lors de la cr�ation de l'utilisateur [voir plus haut].
~\\
D'autres commandes int�ressantes permettent de v�rifier le bon fonctionnement des quotas (quotacheck), d'envoyer des messages d'avertissement aux utilisateurs d�passant la limite douce (warnquota)  et de g�n�rer des statistiques:\\
\texttt{\# repquota -ugva}\\
~\\

Note :\\
Attention, l'utilitaire dd semble mal g�rer les quotas.

Liens : \\
\url{http://www.freenix.fr/unix/linux/HOWTO/mini/Quota.html}\\
\url{http://linux.developpez.com/cours/securedeb/?page=page5\#L5.4}

\section{Crontab}
~\\
Cron est d�mon qui permet d'ex�cuter automatiquement des commandes ou des scripts � une date et une heure sp�cifi�es � l'avance.\\
~\\
C'est �videmment tr�s utile pour toutes les t�ches d'aministration. Le d�mon cron se base sur le fichier \textit{/etc/crontab} pour lancer des actions toutes les heures (cron.hourly), tous les jours (cron.daily), toutes les semaines (cron.weekly) et tous les mois (cron.monthly). \\
~\\
D�j�, il peut �tre int�ressant de personnaliser les param�tres du \textit{/etc/crontab} afin d'�viter l'utilisation des horaires par d�faut.\\
Ensuite on peut placer des scripts (ex�cutables) dans les cron.*ly ou bien dans le r�pertoire cron.d afin de personnaliser l'horaire :
~\\
\textbf{Squelette} :
~\\
"minute" "heure" "jours" "mois" "jour/semaine" "utilisateur" "commande"\\
~\\
Exemple signifiant du lundi au vendredi, toutes les 3h � la 5�me minute:\\
\texttt{5 */3 * * 1-5 root ntpdate serveur-ntp\\
}

\section{Gestion des Journaux}

Les journaux sont des fichiers qui contiennent des informations d'activit� dat�e. Ils sont essentiels pour un serveur pour de nombreuses raisons : v�rifier des actions pass�es, g�n�rer des statistiques, d�boguer un programme. La v�rification des actions pass�es est notamment importante en cas de probl�me (piratage, service d�fectueux). La justice oblige �galement � conserver certains journaux pendant une certaine dur�e. Un flou concerne ce qu'il faut r�ellement conserver (apparemment seules les informations d'ent�tes mais pas le contenu en lui-m�me) et la dur�e (cela varie entre 3 mois, 1 an et 3 ans si l'on se base sur les lois fran�aises ou europ�ennes). Des d�crets d'application devraient �claircir ces points dans les prochains mois.\\
~\\
Sous Debian, les journaux se trouvent g�n�ralement dans le r�pertoire \textit{/var/log}. On va distinguer les journaux syst�mes et les journaux applicatifs. Les journaux syst�mes sont g�r�s par le d�mon SYSLOG\footnote{\url{ftp://ftp.rfc-editor.org/in-notes/rfc3164.txt}}. Sa configuration se trouve dans le fichier \textit{syslog.conf}. Voici quelques ligne extraites de ce fichier~:
\begin{verbatim} 
daemon.*                              -/var/log/daemon.log
kern.*                                -/var/log/kern.log
mail.*                                -/var/log/mail.log
*.=debug;auth,authpriv.none;news.none;mail.none  -/var/log/debug
*.emerg                               *
*.*;auth,authpriv.none                /dev/tty8
\end{verbatim}

Voici les fichiers pincipaux g�n�r�s par SYSLOG : \\
~\\
\begin{itemize}
\item[\textbf{auth.log :}] authentification syst�me (login, su, getty)
\item[\textbf{daemon.log :}] relatif aux daemons
\item[\textbf{mail.* :}] messages relatifs aux mails
\item[\textbf{kern.log :}] messages g�n�r�s par le noyau
\item[\textbf{user.log :}] message g�n�r� par des programmes utilisateur
\item[\textbf{debug :}] messages de bogues
\item[\textbf{messages :}]  messages d'info
\item[\textbf{syslog :}]  tous les messages
\end{itemize}
~\\
~\\
Les journaux applicatifs sont g�n�r�s par chaque application. Ils sont souvent dans un r�pertoire du nom de l'application situ� dans \textit{/var/log}.\\
~\\
L'un des points essentiels est la rotation des journaux, c'est-�-dire l'action de fermer le journal actuel (et �ventuellement le compresser) et d'en ouvrir un autre. Il existe actuellement deux programmes qui se chargent d'effectuer ce travail. Le script \texttt{savelog} (un outil Debian) et le programme \texttt{logrotate}. Par d�faut \texttt{savelog} g�re les journaux syst�me et \texttt{logrotate} g�re les journaux applicatifs (apache, mysql, ppp, etc.).\\
Logrotate est ex�cut� tous les jours (cron.daily) \\
Savelog est ex�cut� tous les jours (crond.daily) pour syslog notamment.\\
Savelog est ex�cut� toutes les semaines (crond.weekly) pour les autres journaux syst�me.\\
~\\
L'option "-d" de savelog permet d'utiliser la date lors de la rotation des journaux et de ne pas les effacer. On pourra donc ajouter ses propres r�gles dans les scripts cron pour faire une sauvegarde distante des journaux (�ventuellement dans une base de donn�es). Pour des serveurs d�di�s (applications cl�s), on peut augmenter la fr�quence des rotations et des sauvegardes distantes, mais �galement utiliser des scripts afin de d�tecter toutes alertes ou anomalies et les envoyer par courrier �lectronique ou m�me SMS. \\
Il existe des programmes analysant les journaux permettant de d�tecter des probl�mes, avoir des statistiques (lire logtool prelude).\\
~\\
Il faut donc bien insister sur la n�cessit� de surveiller les services en production � l'aide d'outils adapt�s. On dispose donc d'outils s'appuyant souvent sur le protocole SNMP (Simple Network Management Protocol) qui permet de g�rer et diagnostiquer les probl�mes en transf�rant des informations syst�me (r�seau, charge, �tat, etc.).
~\\
Pour exploiter ces informations, on pourra tracer des courbes MRTG\footnote{\url{http://people.ee.ethz.ch/~oetiker/webtools/mrtg/}} ou RRDtool\footnote{\url{http://people.ee.ethz.ch/~oetiker/webtools/rrdtool/}}. Ces logiciels permettent de produire des courbes totalement personnalis�es pour tracer des courbes.\\
~\\
%TODO
%http://www.linpro.no/projects/munin/
%http://stat.trackfire.net/statspppd/src/
%http://people.ee.ethz.ch/~oetiker/webtools/smokeping/

\section{OpenSSH}

SSH signifie Secure SHell (Shell s�curis�). Le protocole SSH est en cours de standardisation par l'IETF\footnote{\url{http://www.ietf.org/html.charters/secsh-charter.html}}. Les outils du protocole SSH sont utilis�s par un nombre croissant de personnes. Ils sont destin�s � s�curiser le login � distance, s�curiser les transferts de fichiers et s�curiser les TCP/IP et X11 forwardings. Il peut automatiquement chiffrer, authentifier et compresser des donn�es transmises.\\
~\\
La principale impl�mentation du protocole est OpenSSH\footnote{\url{http://www.openssh.org}}. OpenSSH est une version libre de la suite d'outils du protocole SSH. De nombreux utilisateurs de telnet, rlogin, ftp et autres programmes identiques ne r�alisent pas que leur mot de passe et leurs donn�es sont transmis de fa�on non chiffr�e. OpenSSH chiffre tout le trafic (mots de passe inclus) de fa�on � d�jouer les �coutes r�seau, les prises de contr�le de connexion, et autres attaques. De plus, OpenSSH fournit toute une palette de possibilit�s de tunnel et de m�thodes d'authentification. OpenSSH \textbf{doit} �tre utilis� � la place de telnet et autres logiciels non-s�rs.\\

Pour la plupart de ses fonctionnalit�s cryptographiques, OpenSSH s'appuie sur la biblioth�que OpenSSL. La suite logicielle OpenSSH inclue les programmes ssh qui remplace telnet et rlogin, scp qui remplace rcp, et sftp qui remplace ftp. De plus sshd, la partie serveur, est inclus ainsi que d'autres utilitaires tels que ssh-add, ssh-agent, ssh-keygen, ssh-keysign, ssh-keyscan, et sftp-server. OpenSSH supporte les protocoles SSH 1.3, 1.5 et 2.0. Int�ressons nous � sa configuration, qui se trouve dans le fichier \textit{sshd\_config} :\\

\begin{verbatim}
Protocol 2
LoginGraceTime 30
PermitRootLogin no
AllowUsers moi admin
ClientAliveInterval 15
ClientAliveCountMax 45
\end{verbatim}
~\\

On peut �galement imposer des restrictions suppl�mentaires dans le fichier pam.d/ssh :\\
\begin{verbatim}
auth required pam_listfile.so sense=allow onerr=fail 
item=user file=/etc/loginusers
\end{verbatim}
~\\
Si des utilisateurs normaux sont destin�s � utiliser SSH, il peut �tre int�ressant de l'installer dans une prison Chroot. Voici quelques liens qui expliquent cette mise en place~: \\

Liens : \\
{\small
\url{http://chrootssh.sourceforge.net/index.php} \\
\url{http://www.debian.org/doc/manuals/securing-debian-howto/ap-chroot-ssh-env.fr.html} \\
\url{http://vince.kerneled.org/files/ssh\_chroot.txt} \\
}
~\\

\section{Transfert de fichiers}
~\\
Le transfert de fichiers par le protocole FTP fait circuler identifiants, mot de passe et donn�es en clair sur le r�seau. Il est vraiment pr�f�rable d'utiliser scp ou sftp, du projet OpenSSH, qui permet de transf�rer des fichiers de fa�on plus s�curis�e. Mais dans certains cas, notamment dans le cas de serveur web mutualis�, il est de coutume d'offrir un acc�s FTP. Il faut donc prendre quelques pr�cautions. Par exemple, avec le serveur ProFTPD\footnote{\url{http://www.proftpd.org/}}, certaines directives du fichier de configuration \textit{proftpd.conf} sont importantes : \\
\begin{verbatim}
DefaultRoot ~ 
DenyFilter \*.*/
\end{verbatim}

On pourra �galement utiliser PAM pour limiter l'acc�s selon les utilisateurs. Dans le fichier \textit{/etc/pam.d/proftpd}~: \\
auth       required     pam\_listfile.so item=user sense=deny file=/etc/ftpusers onerr=succeed

~\\
On aura donc un fichier \textit{/etc/ftpusers} qui sp�cifiera les identifiants ne pouvant pas utiliser les services FTP (on pourrait faire l'inverse aussi). \\
~\\
Dans le but de chiffrer les connexions FTP, on peut �galement utiliser une couche SSL. Les serveurs Pure-FTP\footnote{\url{http://www.pureftpd.org/}} ou linux-ftpd-ssl\footnote{\url{http://freshmeat.net/projects/linux-ftpd-ssl/}} permettent cela. Il faut bien noter qu'il faut utiliser un programme client capable de l'utiliser (mais c'est le cas de plus en plus de clients). \\

\section{Authentification}

Le fichier \texttt{/etc/passwd} contient la liste des utilisateurs avec le mot de passe chiffr�. Ainsi lors de la proc�dure d'authentification d'un utilisateur, le syst�me teste si le chiffrement du mot de passe entr� correspond au mot de passe chiffr� (\texttt{/etc/passwd} est accessible en lecture aux utilisateurs). Il est souhaitable d'utiliser l'algorithme de chiffrement SHA-512 pour chiffrer les mots de passe. Il est �galement conseill� d'utiliser l'authentification shadow. Avec cette authentification, les mots de passe du fichier /etc/passwd sont remplac�s par 'x' et sont stock�s dans le fichier \texttt{/etc/shadow}, inaccessible en lecture aux utilisateurs. D'autres m�thodes d'authentification locale existent comme sous OpenBSD\footnote{\url{http://openbsd.org/}}  mais l'utilisation de la m�thode shadow est r�pandue sur les distributions Linux. Il faut bien avoir conscience que cela repose sur la solidit� de l'algorithme de chiffrement\footnote{\url{http://evolix.org/man/crypt.html}}.\\

Le fichier \texttt{/etc/group} stocke la liste des groupes, c'est-�-dire des entit�s regroupant plusieurs utilisateurs et permettant de donner � ce groupe d'utilisateurs les m�mes droits sur des fichiers. Un utilisateur peut conna�tre les groupes auxquels il appartient en tapant la commande \texttt{groups} ou encore \texttt{id}.\\ 
Le fichier /etc/adduser.conf contient les valeurs par d�faut pour les programmes \textit{adduser} \textit{addgroup} \textit{deluser} et \textit{delgroup}. Chaque option est de la forme option = valeur. Les simples ou doubles guillemets sont autoris�s autour de la valeur. Les lignes de commentaires doivent avoir un caract�re di�se (\#) au d�but de la ligne.\\
~\\
Exemple~:\\
\texttt{
DSHELL=/bin/bash\\
DHOME=/home\\
SKEL=/etc/skel\\
QUOTAUSER="forquota"\\
DIR\_MODE=0755\\
}\\
~\\
Le r�pertoire /etc/skel/ contient le profil par d�faut qui sera copi� dans le r�pertoire personnel d'un nouvel utilisateur. Il contient souvent les fichiers .bashrc, .bash\_profile, etc. Pour configurer un profil pour les utilisateurs, on ajoutera des fichiers dans /etc/skel (par exemple des boites � mail, param�tres pour certaines applications, etc.)\\
~\\
La gestion des utilisateurs se fait gr�ce aux commandes adduser et addgroup. Notez bien que les commandes useradd, groupadd, userdel et groupdel ont une syntaxe diff�rente et n'utilisent pas forc�ment les m�mes configurations.\\
~\\
Exemple~:\\
\begin{verbatim}
# addgroup --gid 107 student
# adduser --home /home/jean --shell /bin/bash  
  --uid 1057 --ingroup student jean
\end{verbatim}

~\\
L'utilisation d'un mot de passe al�atoire contenant des lettres majuscules et minuscules, des chiffres et des caract�res sp�ciaux (au total plus de 8 caract�res) est fortement recommand�e.\\ 
Toute trace �crite devra �tre bannie si possible surtout pour un administrateur - c'est son boulot ;) - � l'exception d'endroits s�curis�s (coffre-fort, banque, etc.) accessibles par des sup�rieurs hi�rarchiques. Une proc�dure en cas d'accident corporel de l'administrateur pourra �tre mise en place.\\
~\\
Pour les mots de passe utilisateurs, il peut �tre utile d'emp�cher les mots de passe trop simples (cracklib) et de faire tourner des crackeurs de mots de passe afin de v�rifier en permanence qu'aucun utilisateur n'a un mot de passe trop simple etc.\\
~\\
Voici quelques programmes de g�n�ration de mot de passe al�atoire disponibles sous Debian~:\\
\texttt{otp} \texttt{apg} \texttt{makepasswd} \texttt{pwgen}\\
~\\
Il existe �galement des outils pour tenter de cracker les mots de passe. L'un des plus connu est le programme \texttt{john} qu'il est int�ressant de faire tourner r�guli�rement afin de d�tecter si des utilisateurs ont des mots de passe trop simples.
~\\

\section{Gestion de l'authentification}

Nous savons comment fonctionne la proc�dure d'authentification [voir plus haut] mais au-del� de l'authentification, il est n�cessaire d'imposer des restrictions au niveau de la proc�dure de login pour �viter des d�sagr�ments. Pour cela on dispose de plusieurs moyens de restreindre les acc�s et imposer d'autres param�tres.\\
~\\
Commen�ons par le fichier \texttt{/etc/login.defs} qui d�finit les param�tres d'authentification. Certains param�tres sont notamment importants~:

\begin{verbatim}
# delai minimum entre deux tentatives de login
FAIL_DELAY 10
# journaliser les tentatives rat�es
FAILLOG_ENAB  yes
# retenir les identifiants iconnus essay�s
LOG_UNKFAIL_ENAB yes
# retenir les tentatives r�ussies
LOG_OK_LOGINS yes
# delai maximum pour authentification
LOGIN_TIMEOUT  60
\end{verbatim}

Le fichier \texttt{/etc/securetty} contient la liste des terminaux sur lesquels la connexion de 'root' est autoris�e. Il est conseill� de mettre ce fichier vide afin de d�sactiver le login de 'root'. Seul  'su' permettra donc de devenir superutilisateur. Faire donc~:

\begin{verbatim}
# sed -e 's/^/#/' /etc/securetty > /tmp/sed
# mv /tmp/sed /etc/securetty
\end{verbatim}

~\\
\textbf{Linux-PAM}
~\\
La plupart des distributions Linux utilisent l'authentification PAM par d�faut.\\
De nombreuses applications utilisent l'authentification PAM et notamment login, su qui sont assez importantes...\\
On v�rifiera qu'une application utilise l'authentification PAM gr�ce � \texttt{ldd}\\
Ex~: 

\begin{verbatim}
ldd $(which su)
\end{verbatim}

\textit{Lien~:}
\url{http://www.kernel.org/pub/linux/libs/pam/}\\
~\\
La configuration se passe dans le r�pertoire \texttt{/etc/pam.d/}\\
Si ce r�pertoire est absent, la configuration pourra avoir lieu dans un fichier unique: \texttt{/etc/ldap.conf} (ignor� sinon)\\
Chaque application est configur�e dans un fichier portant son nom.\\
Le r�pertoire \texttt{/etc/security/} contient des fichiers de configuration s�curit� pour PAM.\\
~\\
Le comportement par d�faut est dans le fichier \texttt{/etc/pam.d/other}
La syntaxe des fichiers est constitu�e de lignes telles que :
"type" "niveau" "module" "arguments"\\
~\\
"type" peut �tre :\\
\begin{itemize}
\item[\textbf{auth}] : authentification
\item[\textbf{account}] : v�rification des types de services autoris�s
\item[\textbf{session}] : t�ches � effectuer avant/apr�s l'acc�s
\item[\textbf{password}] : m�canismes d'authentification
\end{itemize}
~\\
"niveau" peut �tre :\\
\begin{itemize}
\item[\textbf{required}] : le succ�s � cette �tape est n�cessaire
\item[\textbf{requisite}] : le succ�s est n�cessaire et l'acc�s est refus� en cas d'erreur
\item[\textbf{sufficient}] : le succ�s � cette �tape suffit
\item[\textbf{optional}] : l'acc�s pourra �tre refus� en fonction d'autres param�tres
\end{itemize}
~\\
"module":\\
(Ils se trouvent par d�faut dans le r�pertoire /lib/security/)\\
\begin{itemize}
\item[\textbf{pam\_access.so}] : restriction d'acc�s avec le fichier access.conf\\
\end{itemize}
~\\
Le fichier \texttt{/etc/security/access.conf} permet de g�rer les permissions de login  (si activ� dans PAM) selon la syntaxe suivante :\\
~\\
~[*] permission : users : origins
~\\
~[*] = + (acc�s permis) ou - (acc�s refus�)\\
~\\
\textbf{Exemples :}
~\\
\begin{itemize}
\item[\textit{-:mechant pasbeau:ALL}] -> les comptes mechant et pasbeau ne peuvent pas se loguer
~\\
\item[\textit{-:ALL EXCEPT admin:ALL EXCEPT LOCAL}] -> seuls admin peut se loguer � distance\\
~\\
Sur un serveur o� seuls les administrateurs peuvent se connecter :
~\\
\item[\textit{-:ALL EXCEPT admin admin2:LOCAL}]
~\\
\item[\textit{pam\_deny.so}] : interdiction d'acc�s
~\\
\item[\textit{pam\_env.so}] : utilise les variables d'environnement de pam\_env.conf en plus du fichier /etc/environment 
~\\
\item[\textit{pam\_filter.so}] : utiliser divers filtres
~\\
\item[\textit{pam\_ftp.so}] : avoir un acc�s de type FTP anonyme
~\\
\item[\textit{pam\_group.so}] : utilise group.conf pour imposer des restrictions selon le groupe
~\\
\item[\textit{pam\_issue.so}] : pour afficher /etc/issue lors d'un login
~\\
\item[\textit{ pam\_lastlog.so}] : donne des informations sur la derni�re connexion de l'utilisateur
~\\
\item[\textit{ pam\_ldap.so}] : module pour l'authentification LDAP
~\\
\item[\textit{ pam\_limits.so }]: restrictions particuli�res avec le fichier limits.conf
~\\
\end{itemize}
~\\
Le fichier \texttt{/etc/security/limits.conf} permet d'imposer des limites diverses sur les groupes ou les identifiants :\\
~\\
Structure: "qui" "type" "quoi" "combien"\\
~\\

\begin{itemize}
\item[\textbf{qui}] = compte, @groupe, *
~\\
\item[\textbf{type}] = soft (soft limits) ou hard (hard limits)
~\\
\item[\textbf{quoi}] = core, data, fsize, memlock, nofile, rss, stack, cpu, nproc, as, maxlogins, priority, locks
\end{itemize}
~\\
\textbf{Exemples }:\\
\begin{verbatim}
ftpusers - maxlogins 3
@invite hard cpu 5
@users hard data 10000
\end{verbatim}
~\\
\begin{itemize}
\item[\textit{pam\_listfile.so}] : permet d'autoriser ou non d'apr�s une liste
~\\
\item[\textit{pam\_mail.so}] : pour indiquer si de nouveaux mails sont arriv�s
~\\
login  session  optional  pam\_mail.so dir=~/Maildir/
~\\
~\\
\item[\textit{pam\_mkhomedir.so}] : pour cr�er le home des utilisateurs authentifi�s (pratique pour les utilisateur NIS ou LDAP)
~\\
\texttt{session required pam\_mkhomedir.so skel=/etc/skel/ umask=022}\\
~\\
\item[\textit{pam\_nologin.so}] : emp�cher tout login ! = root si le fichier /etc/nologin existe
~\\
\item[\textit{pam\_permit.so}] : autoriser l'acc�s (dangereux)
~\\
\item[\textit{pam\_rootok.so}] : autoriser si l'utilisateur est root (id=0)
~\\
\item[\textit{pam\_securetty.so}] : permettre l'acc�s de root seulement si le PAM\_TTY figure dans le fichier \texttt{/etc/securetty}
~\\
\item[\textit{pam\_shells.so}] : permettre l'acc�s si le shell de l'utilisateur est list� dans le fichier  \texttt{/etc/shells}
~\\
\item[\textit{pam\_tally.so}] : permet de bloquer les tentatives au bout d'un certain nombre d'�checs
~\\
\texttt{account required /lib/security/pam\_tally.so per\_user deny=5}\\
~\\
\item[\textit{pam\_time.so}] : permet de restreindre certains services selon le temps d'apr�s le fichier time.conf 
~\\
\texttt{games ; * ; !waster ; Wd0000-2400 | Wk1800-0800}\\
~\\
\item[\textit{pam\_unix.so}] : module standard d'authentification Unix
~\\
\item[\textit{pam\_userdb.so}] : module d'authentification sur une Berkeley DB
~\\
\texttt{auth  sufficient   pam\_userdb.so icase db=/etc/id.db}\\
~\\
\item[\textit{pam\_warn.so}] : journalise certains param�tres pour syslog
~\\
\item[\textit{pam\_wheel.so}] : permettre l'acc�s root uniquement au membre de whell (gid=0)
~\\
\item[\textit{pam\_cracklib.so}] : v�rifie que le mot de passe r�pond � certains crit�res (taille, simplicit�) 
~\\
\texttt{password required pam\_crackedlib.so type=Evolix retry=0 minlen=8}\\
\texttt{dcredit=2 ucredit=2 lcredit=2 ocredit=1}\\
\texttt{password  required pam\_pwdb.so use\_authtok nullok md5}\\
\end{itemize}
~\\
Lien: \url{http://perso.wanadoo.fr/alexandre.vidal/pam/}
~\\

On rappelle que pour avoir des renseignements sur l'utilisateur actuellement connect�, on fera~:\\
\textit{ \# whoami}\\
\textit{ \# id}\\
~\\
On peut �galement avoir divers renseignements sur les utilisateurs~:\\
~\\
\textit{ \# who} : donne les utilisateurs actuellement connect�s\\
\textit{ \# last} : donne les derni�res connections gr�ce � fichier wtmp\\
\textit{ \# w.procs} : donne les utilisateurs actuellement connect�s et des renseignements compl�mentaires\\
~\\

\section{S�curit�}

La s�curit� informatique englobe non seulement la s�curit� r�seau pour se
pr�munir d'attaque locale ou ext�rieure, mais �galement la s�curit� physique,
la gestion des utilisateurs, des droits, des journaux et des sauvegardes.\\

\textbf{S�curit� physique au niveau des infrastructures} \\
~\\
M�me si les gens ont plut�t tendance � l'oublier, la s�curit� physique de la
machine est tr�s importante. Il est n�cessaire qu'un disque dur ne soit pas en
�vidence, pr�t � �tre vol� avec toutes les informations qu'il contient. Il est
important que l'alimentation d'un serveur soit prot�g�e, et qu'il ne suffise
pas juste au pirate de d�brancher le cordon � la prise pour que des dizaines de
couches de s�curit� logicielles soient an�anties.\\

\textbf{S�curit� physique au niveau de l'utilisation} \\
~\\
Lorsque vous tapez un mot de passe, il n'est pas superflu de vous assurer que
personne ne soit pench� ou dessus de votre �paule, ou qu'aucun �l�ment d'�coute
a �t� rajout� (keylogger logiciel ou physique).\\

\textbf{Au niveau du BIOS} \\
~\\
Une premi�re m�thode de s�curisation consiste � autoriser uniquement le disque
dur � d�marrer. Il faut aussi d�finir un mot de passe au niveau du BIOS
emp�chant de changer les param�tres de ce dernier pour un d�marrage sur un
autre p�riph�rique dans l'espoir de pouvoir ensuite avec diff�rents outils
acc�der aux donn�es du disque dur.\\

Il est �galement possible d'avoir des protections complexes, comme une
authentification mutuelle entre le BIOS et le disque, mais ces protections sont
rarement mises en place.\\

\textbf{Au niveau du Boot Loader}\\
~\\
\emph{Lilo}\\
Le fichier de configuration principal est lilo.conf (\textit{/etc/lilo.conf})\\
Dans ce fichier la possibilit� est donn�e de mettre en place un mot de passe par l'ajout de la ligne : \texttt{password = motdepasse}.
A remarquer que contrairement � grub que l'on va d�crire, une modification de lilo.conf pour �tre prise en compte doit �tre suivie de la commande \texttt{\#lilo} ex�cut�e en tant que root.\\
~\\
\emph{Grub}\\
De m�me que pr�cedemment il est possible avec grub d'ajouter un mot de passe qui peut m�me �tre crypt�~:\\
\\
Lancer le shell GRUB~:\\
~\\
\texttt{\# grub}\\
\texttt{grub> md5crypt}\\
~\\
entrez votre mot de passe :\\
\\
Password: *********\\
Encrypted: {\small{\$1\$gxRBf0\$pe0rH7/nG9KPJLvc.XV7V.}}\\
~\\
puis, copiez le mot de passe crypt� dans votre fichier de configuration \texttt{/boot/grub/menu.lst}~:
\begin{verbatim}
password --md5 $1$gxRBf0$pe0rH7/nG9KPJLvc.XV7V.
\end{verbatim}
L'argument password peut �tre utilis� pour restreindre certaines entr�es; dans ce cas il est ins�r� juste sous la ligne "title" �  d�marrer. \\
~\\
Dans le fichier \textit{/etc/inittab} permet � un utilisateur physique de red�marrer la machine. Vous pouvez donc supprimer cette ligne~:\\
\texttt{ca:12345:ctrlaltdel:/sbin/shutdown -t1 -a -r now}
~\\
Pour demander � init (le processus d'initialisation) de r�examiner \texttt{/etc/inittab}, faire~:\\
\texttt{\# init q} \\
~\\
N�anmoins, malgr� toutes ces protections possibles, il faut consid�rer qu'avoir
un acc�s physique � une machine permet d'en faire ce que l'on veut. C'est
pourquoi on se contente souvent des protections basiques comme fermer la salle
des machines � cl�, mettre en place des cadenas sur les boitiers ou encore
avoir un syst�me de vid�osurveillance. \\

\textbf{Firewall}
~\\
Sous Linux, il existe un programme puissant qui permet d'avoir des r�gles de firewall complexes. Ce programme se nomme (depuis le noyau Linux 2.4.x) Netfilter/IPTables\footnote{\url{http://www.netfilter.org/}}. Il permet d'�crire des r�gles en ligne de commande, et donc le lancement de r�gles se fait souvent sous forme de script.\\
~\\
Un script d�taill� se trouve en annexe � titre d'exemple.\\
~\\

%TODO
%Fire-Starter : http://www.fs-security.com/
%http://qtables.radom.org/index.php

\textbf{D�tection d'intrusion}
~\\

Au-del� des pr�cautions prises, il faut avoir conscience des diverses possibilit�s d'intrusion et d'attaque. Dans le but de d�tecter et pr�venir ce type de danger, il existe des programmes de d�tections d'intrusion (IDS) qui permettent d'obtenir des rapports de trafic, des alertes et �ventuellement des r�actions dynamiques.\\
Les logiciels Prelude\footnote{\url{http://www.prelude-ids.org/}} et Snort\footnote{\url{http://www.snort.org/}} sont les IDS libres les plus connus. \\

Ils peuvent �tre coupl�s avec une base de donn�es (MySQL, PostGreSQL) et respectivement avec les logiciels Piwi et ACID (Analysis Console for Intrusion Databases)\footnote{\url{http://acidlab.sourceforge.net/}} pour obtenir les rapports par une interface web.\\
~\\
Il existe un autre type de logiciels de s�curit� : le \textit{pot de miel}. Il s'agit de serveur (ou serveur virtuel) laiss� sous surveillance �troite sur un r�seau et dont le r�le est d'�tre attaqu� et compromis afin d'�tudier le comportement et les outils des pirates.\\
Le programme libre le plus connu est honeyd\footnote{\url{http://www.honeyd.org/}} mais il existe d'autres impl�mentations (tinyhoneypot\footnote{\url{http://www.alpinista.org/thp/}}, etc.) D'autres programmes moins �labor�s permettent de surveiller des actions d'utilisateurs:\\
~\\
TTYSnoop\footnote{\url{http://freshmeat.net/projects/ttysnoop/}} permet de d�river des terminaux de connexion, \\
Snoopy\footnote{\url{http://sourceforge.net/projects/snoopylogger/}} pour intercepter les commandes sur le syst�me, \\
~\\

%TODO
%port-sentry
%ScanLogd : http://www.openwall.com/scanlogd/
%debian package AIDE
On citera �galement les nombreux outils qui permettent d'�couter et d'analyser le trafic r�seau:\\
~\\ 
\texttt{tcpdump, ethereal, dsniff, sniffit\footnote{\url{http://reptile.rug.ac.be/~coder/sniffit/sniffit.html}}, ngrep, hunt, tcpick}\\
\texttt{nast, karpski, vnstat, ndiff, etc.} \\
~\\
\textbf{Piratage ?}
~\\
On peut v�rifier l'int�grit� du syst�me gr�ce � des outils de d�tection des rootkits : Chkrootkit\footnote{\url{http://www.chkrootkit.org/}} :\\ ou Rootkit Hunter\footnote{\url{http://www.rootkit.nl/projects/rootkit\_hunter.html}}\\
~\\

% voir script checksecurity (Debian package)

\textbf{Isoler les services}
~\\

\textit{Chroot} \\
Chroot est un utilitaire permettant d'emprisonner un service dans une arborescence limit�e. Cela permet de limiter l'acc�s aux services offerts et �viter d'avoir un acc�s complet au syst�me en cas de failles logiciel. Sous Debian, il existe quelques services tournant dans une prison Chroot par d�faut (Postfix notamment) mais si l'on veut emprisonner des services comme Bind, Apache, serveur FTP, OpenSSH, il faudra cr�er la prison Chroot manuellement. Le script makejail\footnote{\url{http://www.floc.net/makejail/}} permet d'aider � la cr�ation de prisons Chroot � l'aide de fichiers de configuration. On notera que les syst�mes *BSD (notamment OpenBSD\footnote{\url{http://www.openbsd.org/}}) sont en avance pour les services d'emprisonnement. De nombreux services sont ainsi emprisonn�s par d�faut. \\
~\\
D'autres programmes permettent d'�muler plusieurs syst�mes sur une seule machine :\\

\textit{Vserver} \\
Linux-VServer est projet libre proposant un patch pour le noyau Linux ainsi qu'un ensemble d'outils pour permettre de lancer plusieurs distributions Linux au dessus d'un seul noyau et partager les ressources. \\
Lien : \url{http://www.linux-vserver.org/}\\
~\\

\textit{UML} \\
UML (User Mode Linux) est un projet libre permettant de lancer plusieurs noyaux Linux comme un simple programme (en mode utilisateur donc) ce qui permet d'avoir des machines virtuelles Linux. Le contenu du disque du syst�me virtuel est stock� dans un seul fichier. \\
Lien : \url{http://user-mode-linux.sourceforge.net/}
~\\

\textit{Vmware} \\
Vmware est un programme propri�taire plut�t r�put� permettant de lancer plusieurs machines virtuelles (MS-Windows, Linux, *BSD) Il existe une version Workstation permettant de lancer des machines virtuelles comme des applications et une version Serveur permettant de lancer � distance des machines virtuelles sur un serveur. \\
Lien : \url{http://www.vmware.com/}
~\\
\textbf{Mises-�-jour logicielles de s�curit�}
~\\
"Nous ne cacherons pas les probl�mes. Nous garderons toujours notre base de donn�es des rapports de bogues ouverte au public. Les rapports que les utilisateurs remplissent en ligne seront imm�diatement visibles par les autres." extrait du Contrat social Debian\\
~\\
Comme pour la plupart des logiciels, la politique de s�curit� de Debian est d'avoir une gestion transparente des failles et des bogues. Plusieurs �tudes ont montr� que cette politique �tait la plus efficace en terme de s�curit�. N�anmoins, cela a pour cons�quence pour l'administrateur d'avoir une r�activit� accrue pour faire le n�cessaire d�s qu'un probl�me est annonc�. Il existe donc une �quipe de s�curit� Debian qui est charg�e de publier des paquets corrig�s pour la version stable de Debian d�s qu'un probl�me est d�couvert. Pour pouvoir profiter de ces paquets mis � jour, il faut avoir la ligne suivante dans son fichier \textit{sources.list}~:\
~\\
\texttt{deb http://security.debian.org/ stable/updates main contrib non-free}
~\\
Le minimum est de s'abonner � la liste de diffusion debian-security-announce. Mais pour aller plus loin, il est conseill� de suivre de pr�s les listes de diffusion des logiciels utilis�s en production et �galement des listes traitant de s�curit� notamment la liste Bugtraq\footnote{\url{http://www.securityfocus.com/archive/1}}. \\
~\\

\section{R�seau}
~\\
Au niveau d'un serveur, il est pr�f�rable d'opter pour une configuration r�seau statique. En effet d�pendre d'un tiers (serveur DHCP) pour des param�tres aussi essentiels que les param�tres r�seau n'est pas prudent. Dans le cas o� l'on voudrait quand m�me avoir un configuration par DHCP, il faudra au minimum une authentification par adresse MAC.
~\\
Tout d'abord, parlons de quelques outils r�seau utiles~:
~\\
\begin{itemize}
\item[\textbf{netstat :}] netstat -a -u
\item[\textbf{lsof :}] lsof -i 4 
\item[\textbf{tcpdump :}] tcpdump -X -i eth2 port 22 > log\_ssh
\item[\textbf{nmap :}] http://www.insecure.org/nmap/ nmap -O -P0 192.168.23.27
\item[\textbf{hping :}] outil g�n�rateur de paquet r�seau
\item[\textbf{scapy :}] outil de manipulation de paquets
\item[\textbf{arping :}] arping -a -S 192.168.3.12 -c 3 -i wlan0 192.168.54.22
\item[\textbf{dsniff :}] dsniff -i eth1
\end{itemize}
~\\
On va donc commencer par contr�ler les services r�seau ouverts sur la machine et les fermer si n�cessaires~:\\
~\\
Sur un syst�me install� depuis peu, on trouve souvent ces services ouverts :\\
~\\
\texttt{\# nmap localhost}
\texttt{
9/tcp~~~open~~discard\\
13/tcp~~open~~daytime\\
37/tcp~~open~~time\\
111/tcp~open~~rpcbind\\
113/tcp~open~~auth\\
}
~\\
discard, daytime et time sont d�marr�s par inetd\\
~\\
Inetd est historiquement un "super-serveur" permettant de configurer plusieurs services. Pour d�sactiver les services pr�c�dents on proc�dera � la commande suivante :\\
\begin{verbatim}
# update-inetd --disable discard,daytime,time,ident
\end{verbatim}
\begin{verbatim}
WARNING!!!!!! /etc/inetd.conf contains multiple entries for
the `discard' service. You're about to disable these entries.
Do you want to continue? [n] y
\end{verbatim}
~\\

\textbf{Le super-d�mon} \\

~\\
Si vous avez l'intention d'utiliser le serveur Inetd, il est pr�f�rable
d'utiliser Xinetd\footnote{\url{http://www.xinetd.org/}}, qui remplace Inetd en
offrant des fonctionnalit�s plus compl�tes notamment en terme de s�curit�. Pour
installer Xinetd, on installera le paquet xinetd tout simplement. Voici un
exemple du fichier de configuration xinetd.conf~:

\begin{verbatim}
service imap3
{

        socket_type         = stream
        protocol            = tcp
        wait                = no
        user                = root
        passenv             =
        server              = /usr/bin/tcpd
        server_args         = /usr/sbin/imapd
}
\end{verbatim}
~\\

On fermera �galement le port 111 en supprimant le programme portmap (utile pour NIS ou NFS notamment):\\
\begin{verbatim}
# aptitude remove portmap nfs-comon
\end{verbatim}
~\\

On supprimera tous les services inutiles.\\
On v�rifie les services TCP ouverts par~:\\
~\\
\texttt{
\# nmap localhost\\
\# netstat -a -t\\
}
~\\
Et les services UDP par~:\\
~\\
\texttt{
\# nmap -sU localhost\\
\# netstat -a -u\\
}
~\\
On peut v�rifier quels sont les processus qui ouvrent les ports par :\\
~\\
\texttt{
\# lsof -i 4\\
\# lsof -i 6\\
}
~\\
~\\
{\small{Note : dans le cas o� l'on utilise une configuration r�seau par DHCP, on ne s'�tonnera pas de trouver le port 68 ouvert en UDP.}}


\section{Monitoring}

Voyons un exemple simple qui trace des courbes en fonctions du temps de r�ponse de \texttt{ping}. \\
On installera les programmes snmpd et mrtg.\\
Dans \texttt{/etc/snmp/snmpd.conf}, ajouter cette ligne:\\
\textit{com2sec readonly default public}\\
Attention, bien veillez � supprimer cette ligne :\\
\textit{com2sec paranoia default public}\\
~\\
\texttt{/etc/mrtg/ping :}
~\\
\begin{verbatim}
#!/bin/sh

P=`ping  -c3 -q google.fr |grep avg|cut -d" " -f4`
MIN=`echo $P|cut -d"/" -f1`
MAX=`echo $P|cut -d"/" -f2`
echo $MAX
echo $MIN

\end{verbatim}
~\\
\textit{\bf mrtg.conf :}
\begin{verbatim}

WorkDir: /var/www/mrtg
Language: French

Target[ping]: `/etc/mrtg/ping`
Options[ping]: nopercent,growright,gauge,noinfo, nobanner
MaxBytes[ping]: 10000
AbsMax[ping]: 10000
YLegend[ping]: Latence
ShortLegend[ping]: ms
Legend1[ping]: Latence max en ms
Legend2[ping]: Latence min en ms
LegendI[ping]: Latence Max:
LegendO[ping]: Latence Min:
Title[ping]: Ping sur Google
PageTop[ping]: <h1>Latence Google.fr</h1>
WithPeak[ping]:wmy
Legend4[ping]: Max de la latence min
Legend3[ping]: Max de la latence max

\end{verbatim}
~\\
Il suffit ensuite de lancer la commande :\\ 
\texttt{\# mrtg /etc/mrtg.conf}\\
~\\
On la placera dans un cron pour obtenir des courbes de statistiques r�guli�res.\\
~\\
On pourra cr�er des courbes de statistiques en s'appuyant sur les nombreux outils disponibles sous Linux (sysstat, smartmontools, etc.)\\
~\\
Il existe de nombreux programmes �volu�s permettant de g�n�rer des courbes et statistiques. Citons Nagios\footnote{\url{http://www.nagios.org/}}, Cacti\footnote{\url{http://www.cacti.net/}} et  Ntop\footnote{\url{http://www.ntop.org/}}. Le plus connu d'entre eux est certainement Nagios qui permet de surveiller de nombreux services (SMTP, POP3, HTTP, NNTP, PING, etc.) mais �galement les ressources (charge processeur, utilisation des disques, etc.). On peut visualiser les r�sultats, historiques des probl�mes, journaux par interface web et obtenir des alertes personnalis�es et �crire ses propres plugins pour des v�rifications sp�cifiques. La mise en place de Nagios (ou d'un �quivalent) pour un nombre de serveurs d�passant la dizaine est fortement conseill�e.\\
~\\
Pour une surveillance syst�me en direct, on peut utiliser GKrellM\footnote{\url{http://www.gkrellm.net/}} en mode client-serveur : chaque serveur fait tourner un serveur gkrellmd et pour surveiller tous les serveurs, on d�marre les clients sur un poste de travail. On peut �galement s�curiser les transmissions en encapsulant le trafic dans un tunnel SSH.
~\\
\url{http://www.debian.org/doc/manuals/securing-debian-howto/index.fr.html}\\
\url{http://entreelibre.com/scastro/debian-secinst/debian-secinst.txt}\\

\section{Scripts shell}
~\\
~\\
En informatique, on distingue les langages compil�s et les langages
interpr�t�s. On peut consid�rer qu'un "script" est un programme �crit dans un
langage interpr�t�. Cela comprendra donc les shells (sh, csh, ksh, tcsh, bash,
pdksh) les outils de manipulation de texte (sed, awk), Perl, Tcl, Ruby et
Python.
~\\
~\\
La plupart de ces langages pourraient m�riter une formation enti�re aussi nous nous concentrons sur quelques fonctionnalit�s int�ressantes du shell :\\
~\\
\begin{itemize}
\item[\bf cut]
Utilitaire qui s�lectionne des sections sur chaque ligne d'un fichier sur la sortie standard.\\
~\\
\texttt{cut -d " " -f 1 fichier}\\
~\\
\item[\bf head]
Utilitaire qui renvoie les premi�res parties d'un fichier sur la sortie standard.\\

\texttt{head -n 7 fichier}\\
\item[\bf tail ]
Utilitaire qui renvoie les derni�res parties d'un fichier sur la sortie standard.\\

\texttt{tail -f fichier}\\
\texttt{tail -n 7 fichier}\\
\item[\bf sort ]
Utilitaire qui trie les lignes d'un fichier sur la sortie standard.\\

\texttt{sort -d fichier}\\
\item[\bf tr] 
Utilitaire pour convertir ou supprimer des caract�res d'un fichier sur la sortie standard.\\

\texttt{echo -e "plop\textbackslash nplop" | tr -d "\textbackslash n"}\\

\item[\bf wc]
Utilitaire qui renvoie le nombre de lignes, fichiers ou octets d'un fichier.\\

\texttt{wc  -l fichier}\\
\item[\bf grep]

Utilitaire renvoyant les lignes correspondant au mod�le indiqu�.\\

\texttt{grep -i HtTp /etc/services}\\

\item[\bf seq]
Utilitaire renvoyant une s�quence de nombres.\\

\texttt{seq 4 3 20}\\

\item[\bf for]
Boucle\\
\texttt{for i in 1 2 5; do echo \$i; done}\\
\texttt{for i in \`ls \*.php\`; do mv \$i \$(\$i\%\%.php).html; done}\\
\end{itemize}



\section{Proc�dures de sauvegarde}
~\\
La mise en place de proc�dures de sauvegarde n�cessite souvent un audit pr�cis pour �valuer le juste milieu entre le niveau de s�curit� et le co�t.\\
Prenons quelques exemples concrets.\\
Il est impensable qu'une entreprise, m�me petite, doive mettre la cl� sous la porte en cas de petite castrophe naturelle (temp�te, foudre, incendie).\\
Des sauvegardes dans un endroit physiquement diff�rent sont donc
obligatoires. � l'inverse, pour une petite entreprise, mettre en place
des sauvegardes sur bande rapatri�e dans un coffre-fort toutes les
heures en fourgon blind� sera probablement disproportionn�. Il faut donc
�valuer chaque risque et le chiffrer. Les param�tres � prendre en compte sont:\\
~\\
- les risques\\
- le co�t de mise en place et le co�t r�gulier\\
- le temps d'administration de la solution\\
- le temps de red�ploiement ou de recherche\\

~\\
Il existe principalement deux types de pr�ventions:\\
~\\
- Pr�vention crash mat�riel\\
La solution est � choisir parmi la solution RAID, les sauvegardes syst�me (disque/bande/p�riph�rique amovible) et sauvegarde des fichiers syst�me.\\
~\\
- Pr�vention erreurs logicielles ou humaines\\
La solution est de faire des sauvegardes r�guli�res.\\

~\\
tar est l'outil le plus utilis� pour la sauvegarde.\\
~\\
Exemple :\\
\begin{verbatim}
tar -czvps --same-owner --atime-preserve backup.tar.gz /rep/
\end{verbatim}


~\\
\emph{Table des partitions :}\\
~\\
\textit{Sauvegarde} : \texttt{dd if=/dev/hda of=NOM\_FIC bs=512 count=1}\\
~\\
\textit{Restauration} : \texttt{dd if=NOM\_FIC of=/dev/hda bs=1 count=64 skip=446 seek=446}\\
~\\
\textit{Partimage}\footnote{\url{http://www.partimage.org/}} :
~\\
\begin{itemize}
\item[Sauvegarde] : \\
\texttt{partimage -z1 -o -d save /dev/hda12 /mnt/backup/sav.partimg.gz}\\
\item[Restauration] :\\
\texttt{partimage restore /dev/hda12 /mnt/backup/sav.partimg.gz}\\
\end{itemize}

~\\
L'outil rsync\footnote{\url{http://samba.anu.edu.au/rsync/}} est tr�s puissant car il permet de mettre � jour en local ou � distance uniquement les fichiers modifi�s.\\
~\\
client rsync <-> serveur rsynd\\
~\\
\begin{verbatim}
22/tcp    open  ssh
873/tcp   open  rsync
\end{verbatim}
~\\
Options int�ressantes :\\
\begin{verbatim}
-v, --verbose
\end{verbatim}
-> plus verbeux
\begin{verbatim}
-a, --archive 
\end{verbatim}
-> mode archive (equivalent to -rlptgoD), ne pr�serve pas les liens hard
\begin{verbatim}
-r, --recursive 
\end{verbatim}
-> visite r�cursive des r�pertoires
\begin{verbatim}
-l, --links 
\end{verbatim}
-> copie les liens symboliques comme liens symboliques
\begin{verbatim}
-p, --perms 
\end{verbatim}
-> pr�serve les permissions
\begin{verbatim}
-o, --owner 
\end{verbatim}
-> pr�serve le propri�taire (root uniquement)
\begin{verbatim}
-g, --group 
\end{verbatim}
-> pr�serve le groupe
\begin{verbatim}
-t, --times 
\end{verbatim}
-> pr�serve les dates
\begin{verbatim}
-S, --sparse 
\end{verbatim}
-> traite les fichiers � trous efficacement
\begin{verbatim}
-C, --cvs-exclude
\end{verbatim}
-> ignore automatiquement des fichiers, comme le ferait CVS
\begin{verbatim}
--delete 
\end{verbatim}
-> efface les fichiers qui n'existent pas du cot� exp�dition
\begin{verbatim}
--partial 
\end{verbatim}
-> conserve les fichiers partiellement transf�r�s
\begin{verbatim}
--progress 
\end{verbatim}
-> affiche la progression
\begin{verbatim}
-z, --compress 
\end{verbatim}
-> compresse les donn�es
\begin{verbatim}
-e ssh
\end{verbatim}
-> utilise ssh
\begin{verbatim}
-D, --devices 
\end{verbatim}
-> pr�serve les devices

~\\
L'astuce supr�me consiste � utiliser des "liens hards" gr�ce � la commande \texttt{cp -al}\\
Un fichier est donc supprim� lorsqu'aucun lien mat�riel ne pointe vers lui :\\
~\\
\begin{verbatim}
mv backup.1 backup.2
cp -al backup.0 backup.1
rsync -e -a --delete source backup.0/
\end{verbatim}

% Copyright (c) 2004-2010 Evolix <info@evolix.fr>
%  Permission is granted to copy, distribute and/or modify this document
%  under the terms of the GNU Free Documentation License, Version 1.2
%  or any later version published by the Free Software Foundation;
%  with no Invariant Sections, no Front-Cover Texts, and no Back-Cover Texts.
%  A copy of the license is included at http://www.gcolpart.com/howto/fdl.html

\chapter{Apache}

\section{Rappel de l'architecture client/serveur}

L'architecture client-serveur\footnote{http://www.faqs.org/faqs/client-server-faq/} se r�sume � la demande de services d'un programme client � un programme serveur. Il s'agit de l'extension logique du partitionnement des logiciels importants en modules donnant la possibilit� de d�veloppement et de maintenance plus ais�s. Les modules "demandeurs" sont appel�s client et les modules appel�s sont appel�s service. Ainsi les diff�rents modules fonctionnent sur des plateformes diff�rentes et appropri�es � leur fonction. Par exemple, les syst�mes de gestion de base de donn�es tournent sur des plateformes logicielles et mat�rielles con�ues pour optimiser les requ�tes, ou les serveurs de fichiers tournent sur des plateformes adapt�es pour la gestion de fichiers.\\
~\\
Le client est donc un programme qui envoie un message � un programme serveur, demandant au serveur un service. Les programmes client sont en g�n�ral constitu�s d'une interface permettant de valider les donn�es entr�es par l'utilisateur et d'un programme permettant de traiter et d'envoyer les requ�tes aux programmes serveur.\\
~\\
Le programme contient donc un certain nombre de facilit�s pour interagir avec l'utilisateur. Ainsi, il acc�de aux ressources locales (�cran, clavier, processeur, p�riph�riques, etc.).\\ Un des �l�ments souvent pr�sent sur une machine de type poste de travail est une interface graphique : GUI (Graphical User Interface).\\ 
Normalement, c'est le Windows Manager qui d�tecte les actions de l'utilisateur, g�re les diff�rentes fen�tres et affiche les donn�es.\\
~\\
Le serveur est un programme qui r�pond aux demandes du client en r�alisant la t�che demand�e. Les programmes serveur recoivent en g�n�ral des requ�tes des programmes client, ex�cutent des requ�tes et mises-�-jour sur une base de donn�es, contr�lent l'int�grit� des donn�es et r�pondent aux programmes clients. Le programme serveur devrait �tre sur une machine ind�pendante sur le r�seau mais souvent plusieurs programmes serveur sont sur la m�me machine et dans  certains cas, la machine h�bergeant le service est un poste de travail. Le programme serveur peut souvent acc�der � des resources locales telles que les bases de donn�es, imprimantes, interfaces et processeur(s).\\ 
~\\

\section{Le protocole HTTP}

\subsection{Diff�rentes versions}

HTTP/0.9 : premi�re version du protocole HTTP, tr�s simple, permettant uniquant une requ�te GET et une r�ponse sans m�ta-donn�es. \\
HTTP/1.0 : ancienne version du protocole HTTP, encore utilis�e par certains logiciels. Le serveur HTTP ferme encore la connexion d�s qu'il a envoy� sa r�ponse. \\
HTTP/1.1 : version la plus r�pandue du protocole HTTP. Elle permet notamment les connexions persistantes, la n�gociation du contenu, et une meilleure gestion du cache.~\\
~\\
\subsection{M�thodes :}
~\\
GET : requ�te d'une ressource \\
HEAD : requ�te uniquement des ent�te d'une ressource \\
POST : envoi de donn�es � une ressource \\
Il existes d'autres m�thodes moins utilis�es (PUT, DELETE, TRACE, CONNECT) \\
~\\
\subsection{Codes d'�tat :}
\begin{itemize}
\item{1xx} : Information  (peu utilis�)
\item{2xx} : Succ�s, notamment le code 200 correspondant � OK
\item{3xx} : Redirection, notamment 301 (d�placement d�fintif) et 302 (d�placement temporaire)
\item{4xx} : Erreur du client, notamment 404 (non trouv�) et 403 (non autoris�)
\item{5xx} : Erreur du serveur, notamment 500 (erreur interne)
\end{itemize}

\subsection{Champs d'ent�te :}
\begin{itemize}
\item{Allow}
\item{Authorization}
\item{Content-Encoding}
\item{Content-Length}
\item{Date}
\item{Expires}
\item{From}
\item{If-Modified-Since}
\item{Last-Modified}
\item{Location}
\item{Pragma}
\item{Referer}
\item{Server}
\item{User-Agent}
\item{WWW-Authenticate}
\item{etc.}
\end{itemize}

\section{Pr�sentation}
~\\
Le logiciel Apache est un serveur HTTP. Apparu en 1995, il est d�riv� de nombreux patches pour le serveur NCSA HTTPD\footnote{http://hoohoo.ncsa.uiuc.edu/}. Compl�tement r��crit, son nom serait tir� officieusement de l'appelation "a patchy server", c'est-�-dire un serveur fait de patches. La version officielle indique que le nom a �t� choisi en l'honneur de la tribu Apache, bien connue pour son sens aigu de la strat�gie guerri�re et pour son endurance. D�s 1996, il devenait le serveur HTTP le plus r�pandu sur Internet et sa popularit� ne cesse de cro�tre car en 1999, il �tait pr�sent sur 57\% des serveurs et en 2004, le chiffre atteind 67\% \footnote{http://news.netcraft.com/archives/web\_server\_survey.html}.\\
La fondation Apache, Apache Software Foundation\footnote{http://www.apache.org/foundation/}, a �t� cr��e en 1999 afin de soutenir le d�veloppement d'Apache mais aussi de nombreux autres projets orient�s web (Jakarta, Spamassassin, etc.).\\
Apache est l'un des logiciels libres - sous licence Apache\footnote{http://www.apache.org/licenses/} souvent cit� en exemple quand on parle des logiciels libres car il est notamment r�put� pour sa s�curit� et sa fiabilit�.\\

Liens : \\
\url{http://www.apache.org/}\\
\url{http://en.wikipedia.org/wiki/Apache\_HTTP\_Server}\\
~\\
On distingue actuellement la version 1.x de la version 2.x qui comprend de nombreuses avanc�es telles qu'une nouvelle API, le support natif de l'IPv6 et la possibilit� d'installation sur des plateformes non UNIX. Apache poss�de �galement de nombreux modules (CGI, Perl, PHP, authentification avanc�e, etc.) offrant des possibilit�s de mise en oeuvre de services complexes.\\

\section{Installation}
\subsection{Compilation}
~\\
Comme la plupart des logiciels libres, il est possible de compiler Apache � partir des sources. Cela permet de compiler uniquement avec les options que l'on a besoin et d'avoir des binaires bien adapt�s � sa machine.\\
Pour la compilation en elle-m�me, on applique donc la proc�dure classique. On va reprendre en d�tail cette proc�dure.\\
~\\
On t�l�charge les sources mais �galement le hash MD5 des sources ainsi que la signature PGP (et les cl�s des d�veloppeurs Apache) de ces sources~:

\begin{verbatim}
$ wget apache\_x.y.z.tar.gz
$ wget apache\_x.y.z.tar.gz.md5
$ wget KEYS
$ wget apache\_x.y.z.tar.gz.asc
\end{verbatim}

On v�rifie le bon d�roulement du t�l�chargement des sources en comparant le r�sultat des commandes suivantes~:
\begin{verbatim}
$ md5sum apache_x.y.z.gz
$ cat apache_x.y.z.tar.gz.md5
\end{verbatim}
~\\

On importe les cl�s des d�veloppeurs Apache et on v�rifie l'int�grit� des sources~:

\begin{verbatim}
$ gpg --import KEYS
$ gpg --verify apache_x.y.z.tar.gz.asc
\end{verbatim}

On peut ensuite d�compresser et d�sarchiver les sources~:

\begin{verbatim}
tar -zxvf apache_x.y.z.tar.gz
cd apache_x.y.z.tar.gz
\end{verbatim}

On prend ensuite connaissance des options qui s'offrent � nous gr�ce � la commande~:
\begin{verbatim}
./configure --help
\end{verbatim}
On distinguera les options d'administration (noms des r�pertoires, chemins des librairies, etc.). Par exemple~:
\begin{verbatim}
--sysconfdir=/etc/apache2 --sbindir=/usr/sbin ; 
\end{verbatim}

Et les options relatives aux fonctionnalit�s, par exemple~:

\begin{verbatim}
--with-mpm=worker --enable-ssl --enable-rewrite 
--enable-cgi --enable-dav-fs --enable-dav
\end{verbatim}

On aura bien s�r besoin de nombreuses librairies de d�veloppement pour compiler Apache (l'�tape suivante sert bien s�r � v�rifier leurs pr�sences). Ensuite, on sp�cifie les options choisies avant de lancer l'�tape de v�rification :\\
~\\
\texttt{./configure [options]}\\
~\\
On compile :\\
~\\
\texttt{make}\\
~\\
Et on proc�de � l'installation :\\
~\\
\texttt{make install}\\
~\\

\subsection{Paquets}
~\\
Les paquets offrent plusieurs avantages sur la compilation � partir des sources. Ils permettent notamment de gagner du temps, et parfois de g�rer les d�pendances. On distinguera les paquets RPM\footnote{http://www.rpm.org/}, DEB\footnote{http://www.debian.org/distrib/packages}, etc.\\
~\\
Par exemple, sur un syst�me Debian~:\\
~\\
\textbf{Pour Apache 2 :}
~\\
\texttt{ aptitude install apache2-mpm-prefork}\\
~\\
\emph{Paquets principaux :}\\
~\\
\begin{itemize}
\item[apache2.2-common :] modules de base, documentations et icones pour Apache
~\\
\textit{Plusieurs choix pour Apache MPM (Multi-Processing Module) :}
~\\
\item[apache2-mpm-worker :] version par d�faut. Adapt� aux serveurs � fort trafic
\item[apache2-mpm-prefork :] impl�mentation "non-threaded" (similaire � l'historique Apache 1.3.x)
\item[apache2-mpm-itk :] similaire au prefork, avec la possibilit� de pr�ciser l'utilisateur et le groupe pour chaque VirtualHost
\end{itemize}
~\\

\emph{D�pendances directes :}\\
~\\
\begin{center}
\begin{tabular}{|c|c|}
\hline
libapr0 :& librairie "Apache Portable Runtime"\\
\hline
openssl :& librairies "Authentication abstraction"\\
\hline
ssl-cert :& surcouche pour g�n�rer des certificats\\
\hline
libldap2 :& librairies OpenLDAP\\
\hline
libgnutls11 :& librairies GNU TLS\\
\hline
libgcrypt11 :& librairies cryptographiques LGPL\\
\hline
libgpg-error0 :& librairie pour erreurs/messages composants GnuPG\\
\hline
liblzo1 :& librairies de compression LZO\\
\hline
libopencdk8 :& Kit "Open Crypto Development"\\
\hline
libtasn1-2 :& librairies structures ASN.1\\
\hline
zlib1g :& librairies de compression gzip\\
\hline
libsasl2 :& librairies SASL v2\\
\hline
\end{tabular}
\end{center}

\section{Configuration}

On v�rifiera sa configuration gr�ce � la commande :\\
~\\
\texttt{apache2ctl configtest}\\
~\\
La configuration d'Apache 2 se trouve dans le r�pertoire /etc/apache2/ \\
La configuration principale est dans le fichier apache2.conf \\

\textit{Note :} Selon les syst�mes (distributions Linux, BSD, etc.), cela peut varier : la commande peut �tre apachectl, le r�pertoire de configuration peut �tre /etc/httpd ou /usr/local/etc/apache22 par exemple, et la configuration peut �tre dans un fichier httpd.conf \\
~\\
Passons en revue quelques options � conna�tre gr�ce � un exemple de fichier de configuration.\\
D�taillons la premi�re partie correspondant � l'environnement et aux modules :\\
~\\
\begin{verbatim}
### Section 1: Environnement

# mode d'execution du serveur : inetd ou standalone
ServerType standalone

# repertoire de configuration
ServerRoot /etc/apache2

# lock and PID file
LockFile /var/lock/apache.lock
PidFile /var/run/apache.pid

# temporisation pdt laquelle Apache attend temps total r�ception requ�te GET
# ou entre r�ception paquets TCP lors d'une requ�te POST ou PUT etc.
Timeout 300

# connexions persistentes
#KeepAlive On
# nombre de requ�tes permises pour une connexion unique
# lorsque la directive KeepAlive est activ�e
#MaxKeepAliveRequests 100
# nombre de secondes pendant lesquelles Apache
# attendra une requ�te post�rieure avant de rompre une connexion.
#KeepAliveTimeout 15

# nombre minimum de processus fils en attente qu'un serveur pourra conserver
#MinSpareServers 5
# nombre maximal de processus fils en attente
#MaxSpareServers 10
# nombre de processus fils cr��s d�s le d�marrage du serveur
#StartServers 5

# nombre limite de requ�tes simultan�es pouvant �tre accept�es par le serveur
MaxClients 150
# nombre limite de requ�tes qu'un processus serveur fils peut tra�ter
MaxRequestsPerChild 100

# modules
LoadModule ...
LoadModule ...
LoadModule ...

# MIME 
<IfModule mod_negotiation.c>
    LanguagePriority fr en da nl et de el it ja pl pt pt-br ltz ca es sv
</IfModule>

AddType application/x-httpd-php .html .php .php3
AddType application/x-httpd-php-source .phps
AddType application/x-tar .tgz
...

# avoir le maximum d'informations (mod_status)
<Location /server-status-0906>
    SetHandler server-status
    Order deny,allow
    Deny from all
    Allow from 127.0.0.1
    Allow from 1.2.3.4
</Location>
ExtendedStatus On
<Location /server-info-0906>
    SetHandler server-info
    Order deny,allow
    Deny from all
    Allow from 127.0.0.1
    Allow from 1.2.3.4
</Location>


# xxx
ReadmeName README
HeaderName HEADER
IndexIgnore .??* *~ *# HEADER* README* RCS CVS *,v *,t

# redirection vers fichiers index (mod_dir)
<IfModule mod_dir.c>
    DirectoryIndex index.html index.htm index.shtml index.cgi index.php index.php3 index.php4
</IfModule>

# R�pertoires utilisateurs (mod_userdir)
<IfModule mod_userdir.c>
    #nom du r�pertoire public
    UserDir public_html
    #root n'a pas de site perso
    UserDir disabled root
</IfModule>

<Directory /home/*/public_html>
    AllowOverride FileInfo AuthConfig Limit
    Options MultiViews Indexes FollowSymLinks IncludesNoExec
    <Limit GET POST OPTIONS PROPFIND>
        Order allow,deny
        Allow from all
    </Limit>
    <Limit PUT DELETE PATCH PROPPATCH MKCOL COPY MOVE LOCK UNLOCK>
        Order deny,allow
        Deny from all
    </Limit>
</Directory>

# navigateurs particuliers (mod_setenvif)
<IfModule mod_setenvif.c>
    BrowserMatch "Mozilla2" nokeepalive
    BrowserMatch "MSIE 4.0b2;" nokeepalive downgrade-1.0 
    force-response-1.0
    BrowserMatch "RealPlayer 4.0" force-response-1.0
    BrowserMatch "Java/1.0" force-response-1.0
    BrowserMatch "JDK/1\.0" force-response-1.0
</IfModule>
\end{verbatim}

D�taillons maintenant la seconde partie~:

\begin{verbatim}
### Section 2: configuration principale

#num�ro du port
Port 80
#utilisateur et groupe propri�taire d'apache
User www-data
Group www-data

#adresse e-mail que le serveur peut inclure dans un message d'erreur
#retourn� au client
ServerAdmin webmaster@domaine.tld

#nom d'hote (sert pour redirection)
ServerName www.example.com
ServerAlias example.com tmp.example.com

#R�pertoire racine du serveur
DocumentRoot /var/www

#configuration par defaut
<Directory />
    #pas d'acces par defaut
    Order Deny,Allow
    Deny from all
    #possibilite de liens symboliques ssi liens et destinations
    #ont meme proprio
    Options SymLinksIfOwnerMatch
    #htaccess desactive
    AllowOverride None
</Directory>

<Directory /var/www/>
    Options Indexes Includes FollowSymLinks MultiViews
    AllowOverride AuthConfig FileInfo
    Allow from all
</Directory>

#HTACCESS si directive AllowOverride
AccessFileName .htaccess
<Files ~ " ^.ht">
    Order allow,deny
    Deny from all
</Files>

# resolution inverse double
#HostnameLookups Off
# Desactive version verbeuse
ServerTokens Prod
# ajoute une ligne contenant ServerName et ServerAdmin
# en bas des pages d'erreurs notamment
ServerSignature On

#CGI
ScriptAlias /cgi-bin/ /usr/lib/cgi-bin/
<Directory /usr/lib/cgi-bin/>
    AllowOverride None
    Options ExecCGI -MultiViews +SymLinksIfOwnerMatch
    Order allow,deny
    Allow from all
</Directory>

# icones
Alias /icons/ /usr/share/apache/icons/
<Directory /usr/share/apache/icons>
    Options Indexes MultiViews
    AllowOverride None
    Order allow,deny
    Allow from all
</Directory>

# forcer
#AddDefaultCharset UTF-8
\end{verbatim}

Revenons sur certaines options :\\
~\\
Dans \textbf{<Directory ...></Directory>} :\\
~\\
\textbf{- Gestion des acc�s}\\
~\\
\url{http://httpd.apache.org/docs-2.0/mod/mod\_access.html}\\
~\\

\begin{verbatim}
Order [option]
Allow/Deny from [nom de domaine, adresse IPv4/v6, r�seau]

   Order Deny,Allow
   Allow from 10.1.0.0/255.255.0.0
   Deny from all
\end{verbatim}

\textbf{- Options :}

Options [options]

\begin{itemize}
\item[\textbf{None}] : rien
\item[\textbf{MultiViews}] :  rediriger les demandes selon les pr�f�rences du navigateur (mod\_negotiated)
\item[\textbf{All}] : toutes les options ci-dessous
\item[\textbf{Indexes}] : lister le r�pertoire si il n'y a pas de fichier index (mod\_index)
\item[\textbf{FollowSymLinks}] : suit les liens symboliques
\item[\textbf{SymLinksIfOwnerMatch}] : suit les liens symboliques ssi liens et destinations ont le m�me propri�taire
\item[\textbf{Includes}] : possibilit� de filtres Server-side (mod\_include)
\item[\textbf{IncludesNOEXEC}] : Includes mais sans scripts ex�cutables
\item[\textbf{ExecCGI}] : l'ex�cution de scripts CGI est permise (mod\_cgi)
\end{itemize}

Possibilit�s de faire +/- [options] par rapport � une directive sup�rieure (r�pertoire contenant ou racine)

\textbf{- AllowOverride :}

AllowOverride [options]

Permet de sp�cifier certains param�tres dans des fichiers .htaccess :
\begin{itemize}
\item[\textbf{AuthConfig}] : pour les directives d'authentification (Auth*, Require, etc.)
\item[\textbf{FileInfo}] : pour les directives de contr�le des types de fichier (DefaultType, ErrorDocument, SetHandler, etc.)
\item[\textbf{Indexes}] : pour les directives d'indexation de r�pertoire (DirectoryIndex, DefaultIcon, etc.)
\item[\textbf{Limit}]: permet de sp�cifier les directives de gestion d'acc�s (Allow, Deny, Order)
\item[\textbf{Options}] : permet de sp�cifier les options d'Options
\end{itemize}
~\\
\textbf{Fichiers .htaccess}
~\\
\url{http://httpd.apache.org/docs-2.2/howto/htaccess.html} \\
~\\

\textit{Exemple de fichier .htaccess :}\\

\begin{verbatim}
AuthUserFile .htpasswd
AuthGroupFile /dev/null
AuthName "Acces reserve"
AuthType Basic
<LIMIT GET POST>
Require valid-user
</LIMIT>
\end{verbatim}

~\\
Voir mod\_auth\\
\textbf{<Location ...></Location>} est similaire � <Directory></Directory> � la diff�rence que les directives sont valables sur les chemins d'URL\\
~\\

Exemple~:

\begin{verbatim}

<Location /status>
SetHandler server-status
Order Deny,Allow
Deny from all
Allow from 192.168.176.53
</Location>

\end{verbatim}
Dans \textbf{<Files ...></Files>} :\\
~\\
Directives portant sur les fichiers.\\
~\\
Exemple :\\
\begin{verbatim}

<Files ~".(mp3|ogg|avi|mpeg)\$">
    Order allow,deny
    Deny from all   
</Files>

\end{verbatim}
~\\
\textbf{<Limit ... ></Limit>} impose des restrictions sur certaines m�thodes du protocole HTTP.
~\\
Exemple :\\
\begin{verbatim}
<Limit PUT DELETE PATCH PROPPATCH MKCOL COPY MOVE LOCK UNLOCK>
    Order deny,allow
    Deny from all
</Limit>

\end{verbatim}
~\\

Par d�faut, la configuration d'Apache est souvent r�partie dans plusieurs
fichiers pour une meilleure gestion. Ainsi, sous Debian, on retrouve
le partitionnement suivant, indiqu� dans le fichier de configuration principal :\\

\begin{verbatim}

Include /etc/apache2/mods-enabled/*.load
Include /etc/apache2/mods-enabled/*.conf

Include /etc/apache2/httpd.conf

Include /etc/apache2/ports.conf

Include /etc/apache2/conf.d/[^.#]*

Include /etc/apache2/sites-enabled/[^.#]*

\end{verbatim}
~\\
~\\
\textbf{Les modules :}\\
~\\
On trouve les r�pertoires mods-available et mods-enabled dans le r�pertoire de configuration. mods-available contient des fichiers NOM.load et NOM.conf : un fichier NOM.load contient la directive permettant le chargement d'un module disponible :\\
\begin{verbatim}

LoadModule /chemin/NOM.so

\end{verbatim}
~\\
Le fichier NOM.conf contient les �ventuelles options de configuration du mod�le, par exemple :\\
\texttt{
<IfModule mod\_NOM.c>\\
	...\\
</IfModule>\\
}
~\\
Pour activer un module, on fait simplement un lien symbolique du fichier NOM.load (et NOM.conf si il existe) vers le r�pertoire mods-enabled.\\
~\\
Exemple :\\
{\small
\texttt{
\# ln -s /etc/apache2/mods-available/ssl.load /etc/apache2/mods-enabled/ssl.load
\# ln -s /etc/apache2/mods-available/ssl.conf /etc/apache2/mods-enabled/ssl.conf
}
}
~\\
\textit{Voir les modules}\\
~\\
-Le fichier \texttt{httpd.conf} :\\
~\\
Utilis� pour les directives suppl�mentaires
(vide par d�faut)\\
~\\
-Le fichier \texttt{ports.conf} :

\begin{verbatim}
Listen 80
Listen IP:80
Listen domain.tld:80
\end{verbatim}

La troisi�me possibilit� est � �viter si possible\footnote{http://httpd.apache.org/docs-2.2/dns-caveats.html}\\
~\\

\subsection{VirtualHost}
~\\
Le terme de VirtualHost se r�f�re � la pratique de faire tourner plusieurs sites Internet sur une seule machine alors que l'utilisateur final ne se rend pas compte que les diff�rents sites tournent physiquement sur la m�me machine.\\
~\\
Apache est capable d'avoir des VirtualHost bas�s sur les adresses IP et sur les noms. Attention, les ports d'�coute sont d�finis avec le param�tre Listen. Les VirtualHost ne font que "rediriger" les requ�tes entrantes.\\
~\\
\begin{verbatim}
NameVirtualHost IP:*
NameVirtualHost *

<VirtualHost 10.1.2.3:>
ServerAdmin webmaster@host.foo.com
DocumentRoot /www/docs/host.foo.com
ServerName host.foo.com
ErrorLog logs/host.foo.com-error_log
TransferLog logs/host.foo.com-access_log
</VirtualHost>
\end{verbatim}
~\\
Exemple complexe :\\
\begin{verbatim}
Listen IP1:80
Listen IP2:8080

NameVirtualHost IP1:80

<VirtualHost IP1:80>
DocumentRoot /www/ip1
ServerName www.name1.tld
</VirtualHost>

<VirtualHost IP1:80>
DocumentRoot /www/ip2
ServerName www.name2.tld
</VirtualHost>

#bas� sur l'IP

<VirtualHost IP2:8080>
DocumentRoot /www/ip3
ServerName www.name3.tld
</VirtualHost>
\end{verbatim}
~\\
� l'int�rieur d'un VirtualHost, on peut sp�cifier de nombreuses directives. Souvent il s'agira de~:\\
\begin{verbatim}
DocumentRoot
ServerAdmin
ServerName
ServerAlias
ErrorLog
TransferLog
LogLevel
CustomLog
ServerSignature
ErrorDocument
Rewrite*
etc.
\end{verbatim}
~\\
Ainsi que le <Directory /></Directory> sp�cifiant les droits par d�faut sur les r�pertoires concern�s (DocumentRoot, script CGI, script Perl, icones, manuel, ...)\\
~\\
Voir dans la documentation, les param�tres pouvant s'appliquer dans un VirtualHost.\\
~\\
Lien:\url{http://httpd.apache.org/docs/vhosts/}\\
~\\
\subsection{Configuration des sites en ligne}
~\\
La configuration d'Apache fonctionne souvent avec des VirtualHost... m�me pour un seul site mis en ligne ! On trouve les r�pertoires \textit{sites-enabled} et \textit{sites-available} dans le r�pertoire de configuration. Par exemple, le fichier \textit{default} :

\begin{verbatim}
NameVirtualHost *
<VirtualHost *>
	ServerName www.example.com
	ServerAlias example.com
        ServerAdmin webmaster@example.com
        DocumentRoot /var/www/

        <Directory />
                Order Deny,Allow
                Deny from all
                Options None
                AllowOverride None
        </Directory>
        <Directory /var/www/>
                Options Indexes FollowSymLinks MultiViews
                AllowOverride None
        </Directory>

        ErrorLog /var/log/apache2/error.log
        LogLevel warn
        CustomLog /var/log/apache2/access.log combined
        ServerSignature On
</VirtualHost>
\end{verbatim}

Pour activer un site, on fait simplement un lien symbolique du fichier dans le r�pertoire sites-available vers le r�pertoire sites-enabled. Par contre, Apache passe en revue les liens du r�pertoire sites-enabled dans l'ordre alphanum�rique/alphab�tique. Il faut donc nommer les liens selon ses pr�f�rences. Ainsi, on cr�ra un lien :
~\\
{\small \# ln -s /etc/apache2/sites-available/default /etc/apache2/sites-enabled/000-default}
{\small \# a2ensite test}
~\\
\section{Modules}
~\\
Un grand nombre de modules sont pr�install�s. On cherchera les paquets des modules suppl�mentaires avec la commande :
\texttt{\\
apt-cache search \textasciicircum libapache2-mod \\
}
~\\
\subsection{mod\_cgi}
~\\
Lien~: \url{http://httpd.apache.org/docs-2.2/mod/mod\_cgi.html}\\
~\\
Ce module permet l'ex�cution de scripts CGI (les scripts CGI peuvent �tre �crits en C, Perl, Shell, etc.).\\
~\\
\textbf{Configuration :}
~\\
\begin{verbatim}
ScriptAlias /cgi-bin/ /usr/lib/cgi-bin/
<Directory /usr/lib/cgi-bin/>
    AllowOverride None 
    Options ExecCGI -MultiViews +SymLinksIfOwnerMatch
    Order allow,deny
    Allow from all
</Directory>
\end{verbatim}
~\\
\textbf{Exemple} :
~\\
date.cgi :
\begin{verbatim}
#!/bin/sh
tmp=`/bin/date`
cat << EndFile
Content-type: text/html

<HTML><HEAD><TITLE>Date du serveur</TITLE></HEAD>
<BODY>
<CENTER>
<H1>La date du serveur est</H1>
$tmp
</CENTER>
</BODY>
</HTML>

EndFile
\end{verbatim}

\subsection{mod\_perl}
~\\
Lien:\url{http://perl.apache.org/}\\
~\\
Ce module permet d'ex�cuter des scripts Perl. Il offre de nombreux avantages par rapport aux scripts CGI en Perl (rapidit�, optimisation, etc.)\\
~\\
\texttt{apt-cache search \textasciicircum libapache perl}
~\\
\subsection{mod\_php4}

\textit{voir PHP}

\subsection{mod\_auth}

Lien:\url{http://httpd.apache.org/docs-2.0/mod/mod\_auth.html}\\
~\\
Exemple :\\
~\\
\begin{verbatim}
AuthUserFile /var/apache/passwd/.htpasswd
AuthGroupFile /dev/null
AuthName "Acc�s reserv�"
AuthType Basic
<LIMIT GET POST>
Require valid-user
</LIMIT>
\end{verbatim}
~\\
Exemple :\\
~\\
\begin{verbatim}
mkdir /etc/apache2/pass
htpasswd -c /etc/apache2/pass/.htpasswd user1

htpasswd /etc/apache2/pass/.htpasswd user2
\end{verbatim}
~\\
Souvent dans un fichier .htaccess\\

\subsection{mod\_proxy}
~\\
Lien~: \url{http://httpd.apache.org/docs-2.0/mod/mod\_proxy.html}\\
~\\
Ce module impl�mente un proxy/cache pour Apache. Il g�re les fonctionnalit�s de proxy pour FTP, CONNECT (pour SSL), HTTP/0.9, et HTTP/1.0.\\

\subsection{mod\_rewrite}
~\\
Lien : \url{http://httpd.apache.org/docs-2.0/mod/mod\_rewrite.html}\\
\begin{verbatim}
RewriteEngine On

RewriteCond

Variables :

RegEx :

^ : d�but
$ : fin
. : tous les caract�res
* : nombre infini de fois

RewriteRule
\end{verbatim}
~\\
Exemple : forcer le nom SERVER\_NAME pour le serveur :
\begin{verbatim}

RewriteEngine On
RewriteLog  "/var/log/apache/rewrite.log"
RewriteLogLevel 3
RewriteCond %{HTTP_HOST} !^%www.domaine.tld$
RewriteRule ^/(.*) http://%{SERVER_NAME}/$1 [L,R]

\end{verbatim}

\subsection{mod\_dav}
~\\
Lien : \url{http://httpd.apache.org/docs-2.0/mod/mod\_dav.html}\\
~\\
mod\_dav et mod\_dav\_fs
~\\
dav\_fs.conf:\\
\begin{verbatim}
DAVLockDB /var/lock/apache2/DAVLock/DAVLockDB
\end{verbatim}
~\\

\subsection{mod\_ssl}
~\\
SSLEngine On\\
~\\
Lien:\url{http://httpd.apache.org/docs-2.0/mod/mod\_ssl.html}\\
~\\

On rappelle la proc�dure de g�n�ration d'un certificat auto-sign�~:\\
~\\

On cr�e une "demande" de certificat en se basant sur des param�tres al�atoires ainsi que sur une cl� priv�e \texttt{privkey.pem} prot�g�e par un mot de passe~:

\begin{verbatim}
$ openssl req -new > demande.csr
\end{verbatim}

Si l'on veut supprimer ce mot de passe de protection (utile dans le cas d'un serveur), on ajoute l'argument \texttt{-out cleprivee.pem} et l'on obtient une cl� priv�e \texttt{cleprivee.pem} non prot�g�e~:

\begin{verbatim}
$ openssl rsa -in privkey.pem -out cleprivee.pem
\end{verbatim}

Enfin, on g�n�re le certificat bas� sur la demande et sign� par la cl� priv�e~:

\begin{verbatim}
$ openssl x509 -in demande.csr -out certificat.pem -req -signkey cleprivee.pem -days 365
\end{verbatim}

~\\
On peut ajouter ensuite les lignes suivantes dans le VirtualHost~:
\begin{verbatim}
#activation SSL
  SSLEngine on
#certificats
  SSLCertificateFile /path/to/certs/certificat.pem
#cle privee
  SSLCertificateKeyFile /path/to/certs/cleprivee.pem
\end{verbatim}
~\\
\textbf{Exemple 1} : avoir un site disponible avec HTTP et HTTPS\\

\begin{verbatim}
ports.conf :
Listen 80
Listen 443

sites-available/default :

NameVirtualHost *:80
<VirtualHost *:80>
...
</VirtualHost>

sites-available/default-ssl :
NameVirtualHost *:443
<VirtualHost *:443>
...
#SSL
  SSLEngine on
#certificats
  SSLCertificateFile /etc/apache2/ssl/certificat.cert
#cle privee
  SSLCertificateKeyFile /etc/apache2/ssl/cle-privee.key
</VirtualHost>

\end{verbatim}
~\\
{\small ln -s /etc/apache2/sites-available/default /etc/apache2/sites-enabled/000-default}\\
{\small ln -s /etc/apache2/sites-available/default-ssl /etc/apache2/sites-enabled/001-default-ssl}\\
~\\
\textbf{Exemple 2} : avoir un site disponible en HTTPS et HTTP redirig� vers HTTPS\\
~\\
\begin{verbatim}
sites-available/default-ssl :
NameVirtualHost *:80
NameVirtualHost *:443

<VirtualHost *:80>
        RewriteEngine On
        RewriteRule ^/(.*) https://webmail.domain.tld/ [L,R]
</VirtualHost>

<VirtualHost *:443>
...
</VirtualHost>
\end{verbatim}
~\\
Lien:\url{http://home.earthlink.net/~fjhirsch/Papers/wwwj/article.html}\\
~\\

\subsection{autres options}

UseCanonicalName : On|Off|DNS (d�faut=On)
Permet de sp�cifier que l'on se r�f�re � l'option ServerName pour d�terminer les variables SERVER\_NAME et SERVER\_PORT

\section{Optimisation}
~\\
Lien:\url{http://httpd.apache.org/docs-2.0/misc/perf-tuning.html}\\
\begin{verbatim}
HostnameLookups off
<Files ~ ".(html|cgi)$">
HostnameLookups on
</Files>
\end{verbatim}
~\\
\texttt{AllowOverride None} : partout o� l'on peut (�vite de chercher .htaccess partout)\\
~\\
\texttt{Options SymLinksIfOwnerMatch} � utiliser le moins possible (pas par d�faut)\\
~\\
DirectoryIndex index.php index.html index.html index.cgi index.pl
~\\

\section{S�curit�}
~\\

Pour am�liorer la s�curit� d'Apache, on peut installer mod\_security~:

\begin{verbatim}
# aptitude install libapache2-mod-security2
\end{verbatim}

Avec le fichier de configuration conf.d/mod-security2.conf resemblant � :

\begin{verbatim}
# enable mod_security
SecRuleEngine On
# access to request bodies
SecRequestBodyAccess On
#SecRequestBodyLimit 134217728
#SecRequestBodyInMemoryLimit 131072
# access to response bodies
SecResponseBodyAccess On
#SecResponseBodyLimit 524288
SecResponseBodyMimeType (null) text/html text/plain text/xml
#SecServerSignature "Apache/2.2.0 (Fedora)"

SecUploadDir /tmp
SecUploadKeepFiles Off

# default action
SecDefaultAction "log,auditlog,deny,status:406,phase:2,t:none"

SecAuditEngine RelevantOnly
#SecAuditLogRelevantStatus "^[45]"
# use only one log file
SecAuditLogType Serial
# audit log file
SecAuditLog /var/log/apache2/modsec_audit.log
# what is logged
SecAuditLogParts "ABIFHZ"

#SecArgumentSeparator "&"
SecCookieFormat 0
SecDebugLog /var/log/apache2/modsec_debug.log
SecDebugLogLevel        0

SecDataDir /tmp
SecTmpDir /tmp

#########
# RULES
#########

# File name
SecRule REQUEST_FILENAME "modsecuritytest1"
# Complete URI
SecRule REQUEST_URI "modsecuritytest2"
SecRule REQUEST_FILENAME "(?:n(?:map|et|c)|w(?:guest|sh)|cmd(?:32)?|telnet|rcmd|ftp)\.exe"
\end{verbatim}

~\\

Afin de s�curiser les requ�tes vers l'ext�rieur, il est recommand� d'installation
un proxy tel que Squid. Voici le fichier squid.conf~:

\begin{verbatim}
# ports
http_port 8888 transparent
icp_port 0

# ACL
acl all src 0.0.0.0/0.0.0.0
acl localhost src 127.0.0.1/255.255.255.255
acl INTERNE src 1.2.3.4/32 127.0.0.0/8
acl Safe_ports port 80          # http
acl SSL_ports port 443 563

acl WHITELIST url_regex "/etc/squid/whitelist.conf"

http_access deny !WHITELIST
http_access allow INTERNE
http_access deny all
\end{verbatim}

Avec le fichier /etc/squid/whitelist.conf suivant :

\begin{verbatim}
http://.*debian.org/.*
http://zidane.evolix.net/.*
http://pub.evolix.net/.*
http://www.kernel.org/.*
http://pear.php.net/.*
http://.*akismet.com/.*
http://.*wordpress.org/.*
http://etc.inittab.org/.*
http://.*twitter.com/.*
http://feeds.feedburner.com/.*
http://feeds2.feedburner.com/.*
http://sync.openx.org/.*
http://oxc.openx.org/.*
http://code.openx.org/.*
http://pc.openx.com/.*
http://api.pc.openx.com/.*
http://bid.openx.net/.*
http://blog.openx.org/.*
http://forum.openx.org/.*
http://www.backports.org/.*
\end{verbatim}

Pour l'activer, on ajoute les regles suivantes dans le firewall :

\begin{verbatim}
#HTTPSITES='0.0.0.0/0'
# Proxy
/sbin/iptables -t nat -A OUTPUT -p tcp --dport 80 -m owner --uid-owner proxy -j ACCEPT
/sbin/iptables -t nat -A OUTPUT -p tcp --dport 80 -d 1.2.3.4 -j ACCEPT
/sbin/iptables -t nat -A OUTPUT -p tcp --dport 80 -d 127.0.0.1 -j ACCEPT
/sbin/iptables -t nat -A OUTPUT -p tcp --dport 80 -j REDIRECT --to-port 8888
\end{verbatim}

\section{Surveillance}
~\\
Apache g�n�re donc les logs selon votre configuration.
G�n�ralement, on retrouvera les erreurs dans error.log et les logs des VirtualHost l� o� on veut :)\\
~\\
\textbf{Analyse des logs :} awstats, webalizer, scanerrlog, webdruid, vlogger\\
~\\
\textbf{Outils :}
~\\
\begin{itemize}
\item[\textbf{ab}] - ApacheBench\\
\texttt{ab -n 5000 -c 100 http://www.domaine.com/index.html}\\

\item[\textbf{siege}] - outil de benchmark semblable � ab\\

\item[\textbf{tsung}] - outil de benchmark tr�s puissant\\

\item[\textbf{apachetop}] - surveillance Apache en temps r�el\\

\item[\textbf{Munin}] - surveillance notamment d'Apache via divers graphes\\

Installer le paquet libwww-perl et configurer mod\_status pour assurer
le bon fonctionnement des courbes.

\item[\textbf{awstats}] - analyse de logs\\

\textit{awstats.VHOST.conf} :\\
\begin{verbatim}
LogFile="/var/log/apache/access.VHOST.log"
SiteDomain="www.VHOST.tld"
Lang="fr"
[...]
\end{verbatim}
~\\
\textit{cron.d/awstats.VHOST}\\
{\small
\begin{verbatim}
30 * * * * root [ -x /usr/lib/cgi-bin/awstats.pl -a -f 
/etc/awstats/awstats.VHOST.conf -a -r /var/log/apache/access.VHOST.log ] 
&& /usr/lib/cgi-bin/awstats.pl -config=VHOST -update >/dev/null
\end{verbatim}
}
\end{itemize}
~\\

%http://www.SERVER.tld/cgi-bin/awstats.pl?config=VHOST

\newpage


\newpage
\chapter{� propos de ce document}
Ce support de formation s'inspire de documentations officielles, de pages
Internet, de livres ou de magazines soumis � des droits d'auteurs. Dans la
mesure du possible, les liens vers les sources ont �t� cit�s.\\
~\\
Copyright (c) 2004-2017 Evolix, Gr�gory Colpart, S�bastien Dubois, Romain Dessort, Arnaud Andr� et Alexandre Anriot\\
~\\

Permission vous est donn�e de copier, distribuer et/ou modifier ce document
selon les termes de la Licence GNU Free Documentation License, Version 1.2 ou
ult�rieure publi�e par la Free Software Foundation ; ce document ne comporte
pas de section inalt�rable. Cette licence est disponible � l'adresse suivante:\\
\url{http://www.gnu.org/copyleft/fdl.html}

\end{document}

% $Id: 2J-Debian-LAMP-Tomcat.2011.tex,v 1.6 2011-06-19 17:20:25 sdubois Exp $
